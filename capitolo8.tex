\section{Magnetismo nella Materia}
L'obiettivo del seguente capitolo è quello di introdurre una serie di proprietà di origine magnetica, quindi parleremo di \textit{Magnetismo nella materia}. A differenza degli aspetti macroscopici studiabili con i mezzi dell'elettrodinamica classica, ci si concentrerà sugli aspetti magnetici microscopici della materia e il loro collegamento col le variabili termodinamiche. Lo studio della materia sotto questo punto di vista ha permesso lo sviluppo di concetti quali: \textit{ordinamento spontaneo}, \textit{transione di fase} e \textit{approssimazione di campo medio}.
Partendo inizialmente dallo studio di paramagnetismo e diamagnetismo sarà possibile introdurre il concetto di \textit{approssimazione di particelle indipendneti} e successivamente passare allo studio delle \textit{correlazioni tra particelle in siti differenti} e quindi avere i mezzi di interpretazione del ferromagnetismo nella materia. Un possibile punto di partenza del discorso è sicuramente il teorema di Bohr - Van Leeuwen che afferma che la magnetizzazione nella materia ${\vet M}$ è sempre nulla. Nonostante questo sono evidenti i dipoli così come sono evidenti i fenomeni di ordinamento magnetico. Si inizi col definire la nomenclatura che verrà usata, per esempio tutto verrà scritto in unità c.g.s. In questo modo il campo di induzione magnetica è definito come
\newl{\boxed{{\vet B} = {\vet H} + 4\pi{\vet M}}}
dove con ${\vet M}$ si vuole identificare il vettore magnetizzazione che non è altro che il vettore di momento di dipolo di ogni singolo atomo, mediato sul volume, quindi una cosa del tipo
\newl{{\vet M} = \frac{{\vet \mu}}{V}.} 
Definisco la suscettività magnetica $\chi_M$ come 
\newl{\chi_M = \left(\frac{\partial M}{\partial H}\right)_{H=0}}
considerando in modo approssimato una relazione lineare tra $\vet M$ ed $\vet H$ e possibile scrivere che $M = \chi_M H$. \`E possibile riconoscere, in funzione al valore di $\chi$ diversi comportamenti della materia:
\begin{itemize}
	\item[1.] \textbf{Diamagnetismo}: caratterizzato da $\chi<0$, il sistema tende a "rifiutare" il campo applicato. Esempio: precessioni di Larmor;
	\item[2.] \textbf{Paramagnetismo}: caratterizzato da $\chi>0$, in questo caso se gli  atomi hanno un loro momento magnetico intrinseco, tendono ad allinearlo in direzione del campo;
	\item[3.] \textbf{Ferromagnetismo}: caratterizzato da fenomeni particolari quali \textit{magnetizzazione spontanea} e \textit{ordinamento}. I dipoli magnetici tendono ad orientarsi spontaneamente. In qeusto caso non vale più l'approssimazione lineare $M = \chi_M H$, in quanto il sistema sarà caratterizzato da transizioni di fase e in generale, da comportamenti abbastanza peculiari.
\end{itemize}
C'è un legame molto forte tra questi fenomeni e la temeperatura, quindi è opportuno fare una trattazione termidinamica. Si parte dall'enunciazione del primo principio:
\newl{dU = dQ - dW.}
Il lavoro per un gnereico sistema immerso in un campo magnetico sarà dato, appunto, in parte da lavoro magnetico, una perte di lavoro meccanico e una parte di assorbimento dovuta all'eventuale scambio di particelle, indicanzo con $\zeta$ il potenziale chimico, 
\newl{dW = \mu\cdot dH + PdV - \zeta dN}
\begin{figure}
	\usetikzlibrary{arrows}
	\centering
	\fbox{
		\begin{tikzpicture}[scale=1,auto=center]
			\newcommand{\midarrow}{\tikz \draw[-triangle 90] (0,0) -- +(.1,0);}
			\draw[->] (0,-1) -- (0,5);
			\draw[->] (-1,0) -- (5,0);
			\node[] at (-0.2,-0.2) {$O$};
			\node[fill,thick,circle, inner sep=0pt, minimum size=0.2cm] at (3,3)  {};
			\draw (1,1) -- node {\midarrow} (2,1);
		\end{tikzpicture}
	}
	\caption{Forze in gioco}
	\label{Lorenz}
\end{figure}
Differenziando l'espressione dell'energia libera si ottiene che
\newl{dF = dU -d(TS) = -SdT -PdV +\zeta dN -\mu\cdot dH} 
e si ottengono tutte le relative grandezze in funzione di $F$ a $T$ $V$ ed $N$ fissati,
\newl{M = -\frac{1}{V}\left(\frac{\partial F}{\partial H}\right)_{T,V,N}\,\,\,\,\,,\,\,\,\,\,\chi = -\frac{1}{V}\left(\frac{\partial^2 F}{\partial H ^2}\right)_{T,V,N} .}
In meccanica statistica la funzione di partizione posso ottenerla anche dalla funzione di partizione canonica, che somma su tutti gli stati microscopici
\newl{F = -k_BT\log Z \,\,\,\,,\,\,\,\, Z= \sum_{stati} e^{-H/k_BT}}
dove $k_B$ è la costante di Boltzmann e H è la classica Hamiltoniana di sistema. In alcuni casi è comodo scegliere il potenziale chimico come una rilevante variabile termodinamica, da cui si ottiene il potenziale \textbf{Gran Canonico}
\newl{\Omega = F -\zeta N = \Omega(T,V,\zeta,H)} 
a cui è associata una funzione di partizione gran canonica
\newl{Z_{\Omega}= \sum_{stati,particelle} e^{(\zeta N -H)/k_BT}}
dove la somma è appunto condotta su tutti gli stati e su tutte le particelle del sistema. Differenziando ottengo
\newl{d\Omega = dF -d(\zeta N) = -SdT -PdV -Nd\zeta - \mu\cdot dH}
da cui si derivano ancora le varibili osservabili del sistema
\newl{M = -\frac{1}{V}\left(\frac{\partial \Omega}{\partial H}\right)_{T,V,\zeta} \,\,\,,\,\,\, N =-\left(\frac{\partial \Omega}{\partial \mu}\right)_{T,V,H}}
Differenziando l'energia libera in funzione al gran potenziale canonico
\newl{\left(\frac{\partial F}{\partial H}\right)_N = \left(\frac{\partial(\Omega+\zeta N)}{\partial H}\right) = \left(\frac{\partial \Omega}{\partial H}\right)_\zeta + \frac{\partial \Omega}{\partial \zeta}\frac{\partial \zeta}{\partial H} + N \frac{\partial \mu}{\partial H}}
Come visto precedentemente la magnetizzazione è definita come una funzione di $\zeta$, quindi per ottenere la suscettività magnetica in funzione al numero di particelle si deve sempre partire dalla scrittura del gran potenziale canonico, rimaneggiarlo, ottenendo che
\newl{\chi = \left(\frac{\partial M(\zeta_{H,N}, H)}{\partial H}\right)_N = \frac{\partial M(\zeta,H)}{\partial H} + \frac{\partial M(\zeta,H)}{\partial \zeta}\frac{\partial \zeta}{\partial H}.}
Nella maggior parte dei casi, il potenziale chimico ha una dipendenza quadratica dal modulo del campo magnetico, quindi possiamo considerare $(\partial \zeta / \partial H) \sim 0$ e la suscettività data direttamente dalla formula lineare
\newl{\chi = \left(\frac{\partial M}{\partial H}\right)_{H=0} = -\frac{1}{V} \left(\frac{\partial^2\Omega}{\partial H^2}\right)_\zeta.}
\subsection{Teorema di Bohr - Van Leeuwen}
Viene ripresa la tesi del teorema BVL e ampliata nei concetti. Come detto precedentemente, dal punto di vista della meccanica statistica classica, non sono ponderati i fenomeni magnetici nella materia, indipendenetemente da T e dall' {\vet H} applicato. Questo lo si può mostrare in modo semplice prendendo il valore medio di un'osservabile generica $F(r_i,p_i)$
\newl{\langle F \rangle = \frac{\int d^3r_i d^3p_i F(r_i,p_i) e^{-H(r_i,p_i)/k_BT}}{\left[\int d^3r_i d^3p_i e^{H(r_i,p_i)/k_BT}\right]} }
in presenza di campo magnetico, è risaputo che $p_i$  trasforma attraverso trasformazioni minimali $p_i \to p_i -(e/c)A(r_i) = p'_i$ mentre $r_i$ non cambia. Riscrivendo l'integrale di prima, la parte di potenziale vettore sparisce, perchè dipende solo dagli $r_i$ quindi ottengo che:
\newl{\langle F \rangle_{\text{senza A}}=\langle F \rangle_{\text{con A}}}
da cui si può dedurre che l'eventuale valore medio dell'osservabile $M$ con $H=0$, è uguale a quello per $H\neq0$ perchè il contributo del campo magnetico sparisce. Quindi ho che $\forall H$ 
la magnetizzazione di un sistema è sempre zero. Affermazione totalmente falsa, l'evidenza macroscopica del ferromagnetismo e delle transizioni di fase sono banali. Si conclude che un medello di meccanica statistica classica, non dice nulla riguardo i fenomeni magnetici. I fenomeni di ordinamento spontaneo sono fenomeni miscroscopici che vanno trattati in modo quantistico, infatti per prima cosa $\left[p,A\right]\neq 0$ ed in più va introdotto il concetto di dipolo magnetico intrinseco puramente quantistico, legato allo spin
\newl{\op{\mu_s} = g_s\frac{e}{2mc}\op{S} = -2\mu_B \op{S} \,\,\,,\,\,\,\op{S} _z=\pm\frac{\hbar}{2} }
\subsection{Hamiltoniana di Interazione Magnetica}
Come detto nel precedente paragrafo, il ferromagnetismo ed i fenomeni di ordinamento scpontaneo sono puramente quantistici, dunque il primo passo consiste nello scrivere l'Hamiltoniana del sistema in studio. Per un sistema di un elettrone immerso in un campo magnetico, si ha 
\newl{\op{H} = \frac{1}{2m}\left(\op{p} -\frac{e}{c}A(\op{r} )\right)^2.}
Per avere invarianza dei campi uso una gauge di Coulomb quindi con $\nabla\cdot{\vet A} =0$, che mi permette di scrivere il potenziale vettore e il campo come 
\newl{{\vet A} = \frac{1}{2} {\vet H} \times \vet{r}\,\,\, \implicaa\,\,\, {\vet H = \bf{\nabla} \times {\vet A} }}
in più, nella gauge di Coulomb ho che $\left[p,A\right]=0$. Riscrivendo l'Hamiltoniana di partenza diventa
\newl{\op{H} = \frac{\op{p} ^2}{2m} + \frac{e^2}{8mc^2}({\vet H} \times {\vet r})^2 - \frac{e}{2mc} \op{p} \cdot ({\vet H} \times {\vet r}) }
Supponendo che il campo magnetico sia diretto solo lungo l'asse $\op{z} $ e usando le seguenti relazioni vettoriali
\newl{
	({\vet H}\times {\vet r})^2 &=& H^2r^2 -({\vet H}\times {\vet r})^2 = H^2(x^2+y^2) \nonumber\\
	p\cdot({\vet H}\times {\vet r}) &=& {\vet H}\cdot {\vet r}\times {\vet p} = {\vet H}\cdot {\vet L} = H_z L_z
}
la parte cinetica dell'Hamiltoniana diventa 
\newl{H_c = \frac{p^2}{2m} + \frac{e^2}{8mc^2} H^2(x^2+y^2) -\frac{e}{2mc}L_zH}
il termine quadrato è dovuto al moto nel piano $x-y$, il termine lineare ha la forma di una energia di interazione tra il campo $\vet H $ e il momento di dipolo magnetico $\mu_l$ associato all'elettrone in orbita. In questo modo è ben distinguibile il comportamento paramagnetico, infatti considerando $l=L/\hbar$ ottengo
\newl{\mu_l = -\frac{\abs{e} \hbar }{2m}l = -\mu_B l}
che permette di scrivere l'Hamiltoniana di interazione 
\newl{H_{int} = -\mu_l H = +\mu_B l_z H = -\frac{e}{2mc}L_z H}
inserendo anche i contributi di interazione con campo elettrico, interazione del campo con lo spin e l'interazione spin-orbita ottengo l'Hamiltoniana totale di un sistema immerso in un campo elettromagnetico, definendo $\omega_c = (eH)/(mc)$ la omega di ciclotrone si ottiene
\newl{\op{H} = \frac{\op{p} ^2 }{2m}  + e\varphi(r)  + \frac{m}{8}\omega_c(x^2-y^2) + \mu_B l_z H + 2\mu_B S_z\cdot H +\xi(r) \op{L} \cdot \op{S}   }
le parti elettostatiche e di spin-orbita sono state aggiunte perchè l'elettrone è all'inteno di un atomo, quindi è soggetto anche a questo tipo di interazioni. \`E possibile ora discutere sui termini presente nell'Hamiltonianao. Il primo termine è quello cinetico, il secondo è quello di interazione con il potenziale elttrostatico, il terzo rappresenta lo sviluppo al prim'ordine dell'interazione col campo magnetico, il quarto rappresenta lo sviluppo al second'ordine sempre dell'interazione col campo magnetico, in quinto rappresenta l'interazione dello spin col campo magnetico ed infine si ha l'interazione spin-orbita. \`E possibile notare che \newl{\left[\frac{m}{8}\omega_c^2 (x^2-y^2) \right]  > 0}
quindi è un contributo repulsivo dimagnetico. Dal punto di vista delle grandezze fisiche, però ha un contributo di $\sim 10^{-10}eV$ che molto più piccolo del contributo dei termini quattro e cinque, che si aggirano sui $\sim 10^{-4}eV$ che rappresentano il contributo paramagnetico. Anche lo spin-orbita è di questo ordine di grandezza, $\sim 10^{-2}eV \to 10^{-4}eV$.
Poichè le energie dei livelli elettronici sono nell'ordine dell'$eV$ per eccitare un elettrone sarebbe necessario un campo magnetico gigantesco. In sostanza non è possibile eccitare gli elettroni tramite campi magnetici, quindi tutti gli effetti che sono legati all'interazione magnetica avvengono tutti nel ground state.
\subsection{Materiali Isolanti}
I materiali isolanti sono caratterizzati da atomi le cui bande sono piene. I numeri quantici utili per studiare questo tipo di sistemi sono $L,S,J$ totali, cioè $L=\sum_i l_i$ ecc... Dato lo stato $\psi_{r,s,j,j_z}$ il contributo paramagnetico è dato da
\newl{ \bra{J,J_z} L_z+2S_z \ket{J,J_z} = \mu_B Hg_LJ_z }
che è l'effetto Zeeman. Quindi per questi materiali la parte di Hamiltoniana di interazione sarà
\newl{\op{H} _{par} = \mu_B Hg_L\op{J} _z .}
Di notevole interesse è notare il fatto che l'intesità della correzione all'energia col contributo paramagnetico è nell'ordine di $10^4eV <k_BT$ ! Quindi sul sistema entrano in gioco anche le fluttuazioni dovute alla temperatura, che essendo dello stesso ordine di grandezza, non possono essere trascurate. \`E necessario quindi effettuare una trattazione del problema utilizzando i metodi della meccanica statistica. Come notato, si ha che $k_BT\sim10^{-2}eV \gg E_M$ quindi è necessaria una trattazione statistica. Il primo passo è quello del calcolo della funzione di partizione
\newl{Z=\sum_{stati}e^{-\frac{H}{k_BT}} = \sum_{-J}^{+J} e^{-\frac{\mu_B H g_L J_z}{k_BT}}}
ponendo $\gamma = (g_L \mu_B H) / (k_B T)$ e $J_z = n-J$ la funzione di partizione diventa
\newl{Z = e^{\gamma J} \sum_{n=0}^{2J}e^{-\gamma n} } 
che è molto facile sommare in quanto è una geometrica 
\newl{Z = e^{\gamma J} \frac{1-e^{-\gamma(2J+1)}}{1-e^{-\gamma}} = \frac{\sinh[\gamma(J+1/2)]}{\sinh(\gamma/2)}}
avendo a disposizione la funzione di partizione è possibile quindi passare alla determinazione di tutte le altre variabili termodinamiche del sistema. Di particolare interesse in questo capitolo è la determinazione della magnetizzazione
\newl{M = -\frac{N}{V}\frac{\partial}{\partial H}\left(-k_B T \ln Z\right) = \frac{Nk_BT}{V} \frac{1}{Z} \frac{\partial Z}{\partial \gamma} \frac{\partial \gamma}{\partial H}}
facendo i conti si arriva all'espressione della magnetizzazione in funzione alla funzione di Brillouin
\newl{M = \frac{N}{V}g_L \mu_B J \mathcal{B}_J\left(\frac{g_L\mu_BJH}{k_BT}\right)}
dove la funzione di Brillouin è una funzione adimensionale data da
\newl{\mathcal{B}_J = \frac{2J+1}{2J} \coth\left(x\frac{2J+1}{2J}\right)- \frac{1}{2J}\coth\left(\frac{x}{2J}\right)}
a temperatura ambiente l'energia magnetica è sempre più piccola dell'energia termica $(\mu_B H \ll k_BT \to x\ll 1)$ quindi è possibile espandere la $\coth(x)$ come  
\newl{\coth(x) = \frac{1}{x} + \frac{x}{3} + \cdots}
in questo modo si ottiene una espressione lineare per $M$
\newl{M = \frac{N}{V} \frac{g_L^2 \mu_B^2 J(J+1)}{3k_BT} H}
Infine si raggiunge l'espressione della suscettività paramagnetica di \textbf{Curie} in cui è visibile la dipendenza dall'inverso della temperatura
\newl{\chi_{para} = \frac{N}{V} \frac{(g_L \mu_B)^2 J(J+1)}{3k_B T}.}
Questo suggerisce che l'atomo può essere trattato come se avesse un suo proprio momento magnetico classico. La legge funziona molto bene per quegli ossidi che hanno la shell $f$ semipiena, mentre per le transizioni nei metalli funziona solo se sono verificate le condizioni $L=0$ e $J=S$. Per temperature molto basse e campi magnetici molto intensi il limite si inverte e si ha
\newl{x\to1 \,\,\,\, \coth(x)\to 1,}
e allo stesso modo si ha che la funzione di Brillouin tende ad $1$, $\mathcal{B}_J \sim 1$ e la magnetizzazione non dipende più da $H$ e si ottiene una magnetizzazione di saturazione
\newl{M=\frac{N}{V}g_L \mu_B J}
Se $J=0$, ho che si annulla tutto, quindi devo passare al secondo ordine perturbativo attraverso la teoria di \textbf{Van Vleck}. Se $J=L=S=0$ allora tutti gli oggetti hanno suscettività diamagnetica. Sperimentalmente si osserva che per ioni con shell $f$ semipiena e $J\neq0$ l'accordo con la teoria è molto buono. Per gli elementi di transizione (shell $d$) la teoria non funziona per nulla. Per i casi in cui $J=0$ i il primo ordine perturbativo è nullo 
\newl{\bra{\psi}\mu_BH(L_z+2S_z)\ket{\psi} = \mu_BHg_s J_z =0,}
devo usare il secondo ordine pertubativo rappresentato dalle oscillazioni di Van Vleck.

\textbf{Alcune cose sul quencing orbitale}

\subsection{Gas di Elettroni Liberi - I Metalli}
Nei metalli, i fenomeni di interazione con il campo elettromangetico sono dovuti per lo più dalla risposta degli elettroni liberi nelle bande di conduzione, al campo applicato. Gli effetti di magnetizzazione nei metalli sono caratterizzati da due coportamenti indipendenti:
\begin{itemize}
	\item Paramagnetismo di Pauli;
	\item Diamagnetismo di Landau
\end{itemize}
il primo dovuto all'allineamento dei dipoli magnetici col campo $\vet{H} $, il secondo 
dovuto alla quantizzazione del moto degli elettroni immersi in un campo magnetico, 
così come descritto dall'hamiltoniama di interazione magnetica. Gli elettroni nelle 
bande dei metalli hanno notevoli gradi di libertà quindi l'idea di trattare gli 
elettroni come un gas è ragionevole e da risultati consistenti. Si calcolerà la 
suscettività paramegnetica di Pauli per un gas di elettroni. La derivazione può avvenire 
per vai statistica o per via variazionale, qui di seguito sarà argomentata la via 
statistica. Consideriamo l'ambientazione del problema in un sistema \textbf{gran 
canonico}, in questo modo la funzione di partizione sarà quella gran canonica data da
\newl{Z_\Omega = \prod_{k_x,k_y,k_z,s_z} \left(1+e^{(\mu-E(k,s_z))/k_BT}\right). }
La produttoria è svolta su tutti i singoli stati degli elettroni con vettor d'onda 
$\vet{k} = \left(k_x,k_y,k_z\right)$ e spin $s_z$ aventi energia $E(\vet{k} ,s_z)$. 
