\section{Richiamo sui solidi}
La materia che si presenta sotto il suo stato di \textit{solido} ha la caratteristica di essere completamente ordinata, gode di simmetria per alcuni angoli particolari e ci sono delle periodicità nella struttura. Sistemi solidi amorfi si conoscono per 0D (così detti \textit{quantum dot} ), 1D e 2D (così detti \textit{quantum well} come per esempio il grafene). Oltre ai sistemi amorfi ci sono i \textit{Quasi Cristalli}. Sono caratterizzati dall'avere simmetrie miste, per esempio parti con celle penatgonali miste a parte con simmetria sferica. Per osservare come la materia è aggregata, quali simmetrie sono presenti ed eventuali correlazioni le osservazioni che si possono fare riguardano le figure di diffrazione costruite facendo diffrangere radiazioni differenti. In funzione al passo del reticolo\footnote{E' vera questa cosa? C'è effettivamente una correlazione tra campione che voglio studiare e la relazione che scelgo per studiarlo? Credo proprio di sì.} si possono studiare diverse immagini di diffrazione costruite per esempio con raggi X($1 KeV \sim 10KeV$), con diffrazione di neutroni ($0,1 eV\sim 1 eV$) oppure strutture superficiali, diffrazione di elettroni ($100 eV$). A temperature prossime a quelle dell'ambiente ($T\sim 300K$) i neutroni hanno un loro momento magnetico. In questo modo, vedendo l'interazione tra i netroni e la mataria posso farmi un'idea sulle sue proprietà magnetiche. In conclusione, se devo studiare le proprietà magnetiche della materia uso neutroni a temperatura prossima a quella ambiente. 
\subsection{Reticolo di Bravais}
Un reticolo generico è possibile definirlo come 
\newl{\vet{R} = n_1 \vet{a} _2   + n_2 \vet{a} _2 + n_3 \vet{a} _3} 
in cui i vettori $\vet{a} _i$ non devono essere complanari. Il reticolo a nido d'ape, per esempio, non è un reticolo di Bravais perchè non è un gruppo puntuale.
bla bla bla (va aggiunta roba sui solidi)
