\section{Dinamica degli elettroni in una banda}
Per descrivere la dinamica degli elettroni, in un certo tipo di potenziale, il primo passo è quello di identificare il modo migliore per descrivere l'elettrone stesso. In questo caso, l'approccio migliore è quello usuale, cioè la descrizione dell'elettrone tramite pacchetto d'onda sufficiente locacalizzato. Con sufficientemente localizzato si intende 
\newl{\psi_{\vet k}({\vet r},t) = \sum_{\vet k} f({\vet k})\exp\left[i\left({\vet k} \cdot {\vet r} - \frac{E({\vet r})}{\hbar}t\right)\right]}
in cui sia soddisfatta la condizione di minima incertezza
\newl{\boxed{\Delta {\vet r} \cdot \Delta {\vet k} \sim 1}.}
La lunghezza tipica su cui è scalato il problema, per quanto riguarda ${\vet k}$, è la dimensione della prima zona di Brillouin cioè $2\pi/a$. Avere un pacchetto d'onda ben definito nello spazio ${\vet k}$ vuol dire quindi avere un $\Delta {\vet k} \ll 2\pi/a$. Ovviamente, per continuare a valere la condizione di minima incertezza, se $\Delta k$ è molto piccolo, $\Delta x$ sarà molto grande, in questo caso sarà molto più grande di $\Delta x \gg a/2\pi$. Nel caso in studio è opportuno avere una buona definizione del pacchetto d'onda in $x$ (perchè?). Definiamo la \textit{velocità di gruppo} del pacchetto d'onda, che sarà data da
\newl{\boxed{\langle v \rangle = \frac{d\omega}{dk}=\frac{1}{\hbar}\frac{\partial E(k)}{\partial k}}}
Se abbiamo definito una velocità per il pacchetto d'onda allora possiamo dire che c'è una corrente che caratterizza il mezzo, la cui dentistà è
\newl{J=-e\int \frac{d^3k}{(2\pi)^3}2_s v(k)f(k)}
Se all'elettrone applico una forza, facendolo uscire dallo stato di equilibrio posso scrivere
\newl{dE = F dl = F \cdot v dt = F_\alpha v_\alpha dt }
La forza che agisce sul singolo elettrone sarà del tipo
\newl{F_\alpha = \hbar \frac{dK_\alpha}{dt}.}
In approssimazione semiclassica, l'equazione di newton per il singolo elettrone è possibile scriverla come
\newl{\frac{dv_\alpha}{dt} = \frac{1}{\hbar} \frac{\partial}{\partial t}\left(\frac{\partial E}{\partial k}\right) = \frac{1}{\hbar}\frac{\partial^2E}{\partial k_\alpha \partial k_\beta} \frac{\partial k_\beta}{dt}}
Osservando come è fatta la forma dell'equazione è possibile identificare le varie parti dell'equazione di newton e quindi definire una massa efficacie come 
\newl{\left(\frac{1}{m^*}\right)_{\alpha,\beta}=\frac{1}{\hbar}\frac{\partial^2 E}{\partial k_\alpha \partial k_\beta }}
che è strettamente in funzione alla forma della banda. La dinamica quindi è data dalla curvatura della banda. Se sono in una dimensione non ho problemi, se sono in più dimensioni posso avere masse efficaci diverse in funzione alla diversa curvatura della banda nelle varie direzioni. L'ordine di grandezza della massa efficace si aggira intorno a $0.01 m_e < m^* < 0.1 m_e$.
\subsection{Oscillazioni di Bloch}



Abbiamo visto che è possibile definire una velocità caratteristica del pacchetto d'onda all'interno del cristallo, associandola alla velocità di gruppo ${\vet v} = (1/\hbar) \partial_kE({\vet k})$. Se agisce una forza esterna sull'elettrone l'equazione del moto è possibile scriverla in forma semiclassica come
\newl{\boxed{a_\alpha =\left(\frac{d{\vet v}}{dt}\right)_\alpha = \left(\frac{1}{m^*}\right)_{\alpha,\beta}{\vet F} _\beta}}
spazio $k$ l'elettrone arriva a $\pi/a$ e riparte in modo periodico da $-\pi/a$. Quello che effettivamente succede è che nello spazio reale si ha un'inversione della velocità in prossimità dei punti di diffrazioni, quindi osservo delle oscillazioni, chiamate appunto oscillazioni di Bloch. Il profilo dell'energia e della velocità risulterà essere come in Fig~\ref{vel:en}.
\begin{figure}
	\centering
	\begin{subfigure}[b]{0.3\textwidth}
		\fbox{
		\begin{tikzpicture}[scale=1,auto=center]
			\draw[->] (-2,2.5) -- (2,2.5);
			\draw[->] (0,0) -- (0,5);
			\node[] at (2,2.25) {$k$};
			\node[] at (-0.5,5) {$E(k)$};
			\draw[dashed] (-1.8,0) -- (-1.8,5);
			\draw[dashed] (1.8,0) -- (1.8,5);
			\draw (-1.8,4.5) parabola bend (0,2.5) (1.8,4.5);
		\end{tikzpicture}
	}
		\caption{Grafico $E(k)$}
	\end{subfigure}
	\qquad\quad
	\begin{subfigure}[b]{0.3\textwidth}
		\fbox{
		\begin{tikzpicture}[scale=1,auto=center]
			\draw[->] (-2,2.5) -- (2,2.5);
			\draw[->] (0,0) -- (0,5);
			\node[] at (2,2.25) {$k$};
			\node[] at (-0.5,5) {$v(k)$};
			\draw[domain=-1.8:1.8] plot (\x,{2*sin(\x*180/1.8) + 2.5 });
			\draw[dashed] (-1.8,0) -- (-1.8,5);
			\draw[dashed] (1.8,0) -- (1.8,5);
		\end{tikzpicture}
	}
		\caption{Profilo della velocità}
	\end{subfigure}
	\caption{Velocità ed energia in funzione di  ${\vet k}$, $v_k = \hbar^{-1} \partial_k E$. Le linee tratteggiate indicano il confine della prima zona di Brillouin}
	\label{vel:en}
\end{figure}
Sperimentalmente la corrente di Bloch non si vede a causa delle impurità del cristallo. Se fosse visibile, la densità di corrente di Bloch avrebbe una forma tipica delle densità di corrente
\newl{{\vet J } = -e \int \frac{d^3k}{(2\pi)^3} 2_s {\vet v_k} f({\vet k}) \label{dens:corr}}
dove $f({\vet k})$ indica la densità di portatori di carica sulla banda. \`E abbastanza intuitivo capire che se $f({\vet k})=1$, cioè se la banda è piena, allora $\vet J = 0$ in quanto il profilo velle velocità sarebbe completo per ogni $\vet k$, integrando tutto nella prima zona di Brillouin, si integra $\vet{v_k}$ che è periodica dispari. Questo fa annullare l'integrale (\ref{dens:corr}). Se la banda è semipiena, ed è presente una certa forza esterna, allora in questo caso, per un cristallo perfetto, si ha ${\vet J} \neq 0$. La forza esterna è applicata inserendo il materiale in un campo elettrico, da questo fatto ecco giustificata la legge di Ohm
\newl{\boxed{{\vet J} = \sigma {\vet E}}}
Un ulteriore caso interessante è quello mostrato in Fig.~\ref{semip:cond} in cui si ha la banda di valenza quasi piena e la banda di conduzione con qualche elettrone.
\begin{figure}
	\centering
		\fbox{
		\begin{tikzpicture}[scale=1,auto=center]
			\draw[->] (-2,2.5) -- (2,2.5);
			\draw[->] (0,0) -- (0,5);
			\node[] at (2,2.25) {$k$};
			\node[] at (-0.5,5) {$E(k)$};
			\draw[dashed] (-1.8,0) -- (-1.8,5);
			\draw[dashed] (1.8,0) -- (1.8,5);
			\draw [black] plot [smooth,tension=0.4] coordinates {(-1.8,1.5) (-1.75,1.5) (-1.3,1.6) (0,2.5) (1.3,1.6) (1.75,1.5) (1.8,1.5)};

			\draw [black] plot [smooth,tension=0.4] coordinates {(-1.8,4) (-1.75,4) (-1.3,4.1) (0,3) (1.3,4.1) (1.75,4) (1.8,4)};

			\draw[->] (3,0) -- (3,5);
			\draw (3,1.2) -- (4,1.2);
			\draw (4,1.2) -- (4,2.5);
			\draw [black] plot [smooth,tension=1] coordinates {(4,2.5) (3.75,2.8) (3.15,3.25) (3,3.6)};	
			\draw[dashed] (0,3) -- (4.5,3);
			\draw[dashed] (-1.2,3.45) -- (4.5,3.45);
			\node[] at (3.5,0) {$f(E)$};
		\end{tikzpicture}
	}
	\caption{Esempio con banda di valenza pieno e banda di conduzione con pochi elettroni}
	\label{semip:cond}
\end{figure}


\begin{figure}
	\centering
		\fbox{
		\begin{tikzpicture}[scale=1,auto=center]
			\draw[->] (-2,2.5) -- (2,2.5);
			\draw[->] (0,0) -- (0,5);
			\node[] at (2,2.25) {$k$};
			\node[] at (-0.5,5) {$E(k)$};
			\draw[dashed] (-1.8,0) -- (-1.8,5);
			\draw[dashed] (1.8,0) -- (1.8,5);
			\draw [black] plot [smooth,tension=0.4] coordinates {(-1.8,1.5) (-1.75,1.5) (-1.3,1.6) (0,2.5) (1.3,1.6) (1.75,1.5) (1.8,1.5)};

			\draw [black] plot [smooth,tension=0.4] coordinates {(-1.8,4) (-1.75,4) (-1.3,4.1) (0,3) (1.3,4.1) (1.75,4) (1.8,4)};

			\draw[->] (3,0) -- (3,5);
			\draw (3,1.2) -- (4,1.2);
			\draw (4,1.2) -- (4,2.5);
			\draw [black] plot [smooth,tension=1] coordinates {(4,2.5) (3.75,2.8) (3.15,3.25) (3,3.6)};	
		\end{tikzpicture}
	}
	\caption{Esempio con banda di valenza pieno e banda di conduzione con pochi elettroni}
	\label{semip:cond}
\end{figure}

In questo caso la densità di corrente di Bloch è possibile scriverla come 
\newl{{\vet J} = 0 = -e \int\frac{d^3k}{(2\pi)^3} {\vet v} (k) f(k\pi) -e\int\frac{d^3k}{(2\pi)^3} {\vet v} (k) (1-f(k\pi) }
da cui si arriva all'ugualianza
\newl{e \int\frac{d^3k}{(2\pi)^3} {\vet v} (k) f(k\pi) = -e\int\frac{d^3k}{(2\pi)^3} {\vet v} (k) (1-f(k\pi) = {\vet J} _{semipiena} }
in cui la corrente nella banda semipiena è uguale alla corrente di buche cambiate di segno. La buca esiste solo nel materiale, non esiste come entit\`a a s\`e, quindi si definisce "quasiparticella". 


