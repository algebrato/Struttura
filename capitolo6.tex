\section{Dinamica degli elettroni in una banda}
Per descrivere la dinamica degli elettroni, in un certo tipo di potenziale, il primo passo è quello di identificare il modo migliore per descrivere l'elettrone stesso. In questo caso, l'approccio migliore è quello usuale, cioè la descrizione dell'elettrone tramite pacchetto d'onda sufficiente locacalizzato. Con sufficientemente localizzato si intende 
\newl{\psi_{\vet k}({\vet r},t) = \sum_{\vet k} f({\vet k})\exp\left[i\left({\vet k} \cdot {\vet r} - \frac{E({\vet r})}{\hbar}t\right)\right]}
in cui sia soddisfatta la condizione di minima incertezza
\newl{\boxed{\Delta {\vet r} \cdot \Delta {\vet k} \sim 1}.}
La lunghezza tipica su cui è scalato il problema, per quanto riguarda ${\vet k}$, è la dimensione della prima zona di Brillouin cioè $2\pi/a$. Avere un pacchetto d'onda ben definito nello spazio ${\vet k}$ vuol dire quindi avere un $\Delta {\vet k} \ll 2\pi/a$. Ovviamente, per continuare a valere la condizione di minima incertezza, se $\Delta k$ è molto piccolo, $\Delta x$ sarà molto grande, in questo caso sarà molto più grande di $\Delta x \gg a/2\pi$. Nel caso in studio è opportuno avere una buona definizione del pacchetto d'onda in $x$ (perchè?). Definiamo la \textit{velocità di gruppo} del pacchetto d'onda, che sarà data da
\newl{\boxed{\langle v \rangle = \frac{d\omega}{dk}=\frac{1}{\hbar}\frac{\partial E(k)}{\partial k}}}
Se abbiamo definito una velocità per il pacchetto d'onda allora possiamo dire che c'è una corrente che caratterizza il mezzo, la cui dentistà è
\newl{J=-e\int \frac{d^3k}{(2\pi)^3}2_s v(k)f(k)}
Se all'elettrone applico una forza, facendolo uscire dallo stato di equilibrio \`e possibile scrivere
\newl{dE = F dl = F \cdot v dt = F_\alpha v_\alpha dt }
La forza che agisce sul singolo elettrone sarà del tipo
\newl{F_\alpha = \hbar \frac{dK_\alpha}{dt}.}
In approssimazione semiclassica, l'equazione di newton per il singolo elettrone è possibile scriverla come
\newl{\frac{dv_\alpha}{dt} = \frac{1}{\hbar} \frac{\partial}{\partial t}\left(\frac{\partial E}{\partial k}\right) = \frac{1}{\hbar}\frac{\partial^2E}{\partial k_\alpha \partial k_\beta} \frac{\partial k_\beta}{dt}}
Osservando come è fatta la forma dell'equazione è possibile identificare le varie parti dell'equazione di newton e quindi definire una massa efficacie come 
\newl{\left(\frac{1}{m^*}\right)_{\alpha,\beta}=\frac{1}{\hbar}\frac{\partial^2 E}{\partial k_\alpha \partial k_\beta }}
che è strettamente in funzione alla forma della banda. La dinamica quindi è data dalla curvatura della banda. Se sono in una dimensione non ho problemi, se sono in più dimensioni \`e possibile avere masse efficaci diverse in funzione alla diversa curvatura della banda nelle varie direzioni. L'ordine di grandezza della massa efficace si aggira intorno a $0.01 m_e < m^* < 0.1 m_e$.
\subsection{Oscillazioni di Bloch}



Abbiamo visto che è possibile definire una velocità caratteristica del pacchetto d'onda all'interno del cristallo, associandola alla velocità di gruppo ${\vet v} = (1/\hbar) \partial_kE({\vet k})$. Se agisce una forza esterna sull'elettrone l'equazione del moto è possibile scriverla in forma semiclassica come
\newl{\boxed{a_\alpha =\left(\frac{d{\vet v}}{dt}\right)_\alpha = \left(\frac{1}{m^*}\right)_{\alpha,\beta}{\vet F} _\beta}}
spazio $k$ l'elettrone arriva a $\pi/a$ e riparte in modo periodico da $-\pi/a$. Quello che effettivamente succede è che nello spazio reale si ha un'inversione della velocità in prossimità dei punti di diffrazioni, quindi osservo delle oscillazioni, chiamate appunto oscillazioni di Bloch. Il profilo dell'energia e della velocità risulterà essere come in Fig~\ref{vel:en}.
\begin{figure}
	\centering
	\begin{subfigure}[b]{0.3\textwidth}
		\fbox{
		\begin{tikzpicture}[scale=1,auto=center]
			\draw[->] (-2,2.5) -- (2,2.5);
			\draw[->] (0,0) -- (0,5);
			\node[] at (2,2.25) {$k$};
			\node[] at (-0.5,5) {$E(k)$};
			\draw[dashed] (-1.8,0) -- (-1.8,5);
			\draw[dashed] (1.8,0) -- (1.8,5);
			\draw (-1.8,4.5) parabola bend (0,2.5) (1.8,4.5);
		\end{tikzpicture}
	}
		\caption{Grafico $E(k)$}
	\end{subfigure}
	\qquad\quad
	\begin{subfigure}[b]{0.3\textwidth}
		\fbox{
		\begin{tikzpicture}[scale=1,auto=center]
			\draw[->] (-2,2.5) -- (2,2.5);
			\draw[->] (0,0) -- (0,5);
			\node[] at (2,2.25) {$k$};
			\node[] at (-0.5,5) {$v(k)$};
			\draw[domain=-1.8:1.8] plot (\x,{2*sin(\x*180/1.8) + 2.5 });
			\draw[dashed] (-1.8,0) -- (-1.8,5);
			\draw[dashed] (1.8,0) -- (1.8,5);
		\end{tikzpicture}
	}
		\caption{Profilo della velocità}
	\end{subfigure}
	\caption{Velocità ed energia in funzione di  ${\vet k}$, $v_k = \hbar^{-1} \partial_k E$. Le linee tratteggiate indicano il confine della prima zona di Brillouin}
	\label{vel:en}
\end{figure}
Sperimentalmente la corrente di Bloch non si vede a causa delle impurità del cristallo. Se fosse visibile, la densità di corrente di Bloch avrebbe una forma tipica delle densità di corrente
\newl{{\vet J } = -e \int \frac{d^3k}{(2\pi)^3} 2_s {\vet v_k} f({\vet k}) \label{dens:corr}}
dove $f({\vet k})$ indica la densità di portatori di carica sulla banda. \`E abbastanza intuitivo capire che se $f({\vet k})=1$, cioè se la banda è piena, allora $\vet J = 0$ in quanto il profilo velle velocità sarebbe completo per ogni $\vet k$, integrando tutto nella prima zona di Brillouin, si integra $\vet{v_k}$ che è periodica dispari. Questo fa annullare l'integrale (\ref{dens:corr}). Se la banda è semipiena, ed è presente una certa forza esterna, allora in questo caso, per un cristallo perfetto, si ha ${\vet J} \neq 0$. La forza esterna è applicata inserendo il materiale in un campo elettrico, da questo fatto ecco giustificata la legge di Ohm
\newl{\boxed{{\vet J} = \sigma {\vet E}}}
Un ulteriore caso interessante è quello mostrato in Fig.~\ref{semip:cond} in cui si ha la banda di valenza quasi piena e la banda di conduzione con qualche elettrone.
\begin{figure}
	\centering
		\fbox{
		\begin{tikzpicture}[scale=1,auto=center]
			\draw[->] (-2,2.5) -- (2,2.5);
			\draw[->] (0,0) -- (0,5);
			\node[] at (2,2.25) {$k$};
			\node[] at (-0.5,5) {$E(k)$};
			\draw[dashed] (-1.8,0) -- (-1.8,5);
			\draw[dashed] (1.8,0) -- (1.8,5);
			\draw [black] plot [smooth,tension=0.4] coordinates {(-1.8,1.5) (-1.75,1.5) (-1.3,1.6) (0,2.5) (1.3,1.6) (1.75,1.5) (1.8,1.5)};

			\draw [black] plot [smooth,tension=0.4] coordinates {(-1.8,4) (-1.75,4) (-1.3,4.1) (0,3) (1.3,4.1) (1.75,4) (1.8,4)};

			\draw[->] (3,0) -- (3,5);
			\draw (3,1.2) -- (4,1.2);
			\draw (4,1.2) -- (4,2.5);
			\draw [black] plot [smooth,tension=1] coordinates {(4,2.5) (3.75,2.8) (3.15,3.25) (3,3.6)};	
			\draw[dashed] (0,3) -- (4.5,3);
			\draw[dashed] (-1.2,3.45) -- (4.5,3.45);
			\node[] at (3.5,0) {$f(E)$};
		\end{tikzpicture}
	}
	\caption{Esempio con banda di valenza pieno e banda di conduzione con pochi elettroni}
	\label{semip:cond}
\end{figure}


\begin{figure}
	\centering
		\fbox{
		\begin{tikzpicture}[scale=1,auto=center]
			\draw[->] (-2,2.5) -- (2,2.5);
			\draw[->] (0,0) -- (0,5);
			\node[] at (2,2.25) {$k$};
			\node[] at (-0.5,5) {$E(k)$};
			\draw[dashed] (-1.8,0) -- (-1.8,5);
			\draw[dashed] (1.8,0) -- (1.8,5);
			\draw [black] plot [smooth,tension=0.4] coordinates {(-1.8,1.5) (-1.75,1.5) (-1.3,1.6) (0,2.5) (1.3,1.6) (1.75,1.5) (1.8,1.5)};

			\draw [black] plot [smooth,tension=0.4] coordinates {(-1.8,4) (-1.75,4) (-1.3,4.1) (0,3) (1.3,4.1) (1.75,4) (1.8,4)};

			\draw[->] (3,0) -- (3,5);
			\draw (3,1.2) -- (4,1.2);
			\draw (4,1.2) -- (4,2.5);
			\draw [black] plot [smooth,tension=1] coordinates {(4,2.5) (3.75,2.8) (3.15,3.25) (3,3.6)};	
		\end{tikzpicture}
	}
	\caption{Esempio con banda di valenza pieno e banda di conduzione con pochi elettroni}
	\label{semip:cond}
\end{figure}

In questo caso la densità di corrente di Bloch è possibile scriverla come 
\newl{{\vet J} = 0 = -e \int\frac{d^3k}{(2\pi)^3} {\vet v} (k) f(k\pi) -e\int\frac{d^3k}{(2\pi)^3} {\vet v} (k) (1-f(k\pi) }
da cui si arriva all'ugualianza
\newl{e \int\frac{d^3k}{(2\pi)^3} {\vet v} (k) f(k\pi) = -e\int\frac{d^3k}{(2\pi)^3} {\vet v} (k) (1-f(k\pi) = {\vet J} _{semipiena} }
in cui la corrente nella banda semipiena è uguale alla corrente di buche cambiate di segno. La buca esiste solo nel materiale, non esiste come entit\`a a s\`e, quindi si definisce "quasiparticella". 
\subsection{Comportamenti oscillatori nei metalli}
La determinazione della superficie di Fermi cio\`e di quella superficie nello spazio $k$ delimitata dalle energie di occupazioni degli stati a bassa temperatura, \`e di particolare importanza nello studio della fisica dei metalli. La forma della superficie di fermi determina lo stato di equilibrio, le sue propriet\`a ottiche, i coefficienti di trasporto ed \`e il banco di prova per confrontare i calcoli teorici sulla struttura delle bande.
Circa le varie metodologie usate per indagare la forma della superficie di Fermi, quella che sfrutta \textit{l'effetto di De Haas - Van Alphen} \`e sicuramente quella pi\`u famosa e dal punto di vista fisico, strettamente collegata alle essenziali propriet\`a quantiche della materia condensata. L'effetto di De Haas - Van Alphen si pu\`o riassumere come \textbf{comportamento oscillatorio della magnetizzazione dei metalli (o semimetalli) come funzione di un campo magnetico applicato}.
L'effetto di De Haas - Van Alphen si presenza in quesi sistemi metallici molto puri, a temperatura molto bassa e tipicamente in presenza di campo magnetico molto forte, di alcuni Tesla. Le oscillazioni della magnetizzazione risultano da un fenomeno conosciuto come \textit{Quantizzazione di Landau} in cui gli elettroni in un metallo possono essere presenti solo solo su una serie di orbitali quantizzati in un campo magnetico. A causa del fatto che il numero di occupazione dei livelli di Landau cambia in funzione al campo magnetico applicato e quindi \`e possibile osservare oscillazioni nella manetizzazione ( o meglio, nella suscettivit\`a magnetica $\chi_M = \partial M / \partial H$ ) come previsto da Landau nel 1930. Queste oscillazioni sono esattamente periodiche con l'inverso del campo magnetico, come previsto da Onsager il quale ha permesso di collegare il periodo delle oscillazioni $\Delta(1/H)$ con l'area della sezione d'urto estremale della superficie di Fermi in un piano ortogonale alla direzione del campo applicato.

Gli esperimenti che indagano l'effetto di De Haas - Van Alphen, sono spesso caratterizzati da misure di voltaggio indotto in sensori arrotolati intorno al campione e a loro volta immersi in un grande solenoide che ne modula il campo magnetico interno. Il voltaggio nel singolo sensore \`e quindi proporzionale alla variazione di magnetizzazione nel tempo, quindi \`e anche un indice sulla suscettivit\`a magnetica. 
\subsection{Teoria semiclassica per elettroni di Bloch in campo magnetico uniforme}
Si consideri inizialmente una descrizione semiclassica della dinamica degli elettroni nelle bande energetiche. Gli elettroni sono rappresentati da pacchetti d'onda, come combinazione lineare di stati Bloch, piccati intorno ad un ben definito vettor d'onda $k$ e di una larghezza in modo tale che il pacchetto d'onda risulti distribuito su pi\`u celle del reticolo cristallino. La risposta all'applicazione di un campo elettrico o magnetico, che varia molto lentamente rispetto alla lunghezza del pacchetto d'onda da considerarlo constante in zone piccole del reticolo, \`e data da due equazioni della dinamica che collegano la velocit\`a dell'elettrone alla curvatura della banda energetica
\newl{\frac{\partial r}{\partial t} &=& v(k) = \frac{1}{\hbar} \frac{\partial E(k)}{\partial k}\\  
\hbar\frac{\partial k }{\partial t} &=& \frac{e}{c} v(k)\times H }
da cui segue immediatamente che l'energia $E(k)$ e la componente  del vettore d'onda $k$ parallelo alla direzione del campo magnetico, sono entrambe costanti del moto. Gli elettroni si muovono quindi lungo curve nello spazio $k$ (del reticolo reciproco) date dall'intersezione tra le superfici ad energia costante e i piani perpendicolari ad H. In particolare \`e possibile notare che si hanno orbite chiuse (non si \`e arrivati ancora al bordo della zona di Brillouin) e orbite aperte (\`e stata toccata la zona di Brillouin) in funzione alla geometria della superficie ad energia costante. In una teoria semiclassica la distribuzione delle orbite e dei livelli energetici \`e un quasi continuo e riflette la geometria delle superfici ad energia costante (questo non \`e totalmente vero in una teoria completamente quantistica).

Un importante risultato circa il moto di elettroni di Bloch in un campo magnetico che sar\`a molto utile \`e una \textit{relazione tra il periodo di un orbita chiusa e il cambiamento dell'area planare racchiusa dall'orbita, in fiunzione all'energia}. Partendo dall'equazione precedente scritta coi moduli
\newl{\abs{\frac{\partial k}{\partial t} } = \frac{|e|H}{\hbar^2c}\abs{\left(\frac{\partial E(k)}{\partial k}\right)_\perp }  }
dove $(\partial E(k)/ \partial k)_\perp$ \`e la componente del gradiente dell'energia perpendicolare al campo, in alre parole la sua proiezione sul piano dell'orbita. Il periodo di un orbita pu\`o essere determinato come 
\newl{T=\oint\,dt = \oint\frac{dk}{|\partial k / \partial t|} = \frac{\hbar^2c}{|e|H} \oint \frac{dk}{|(\partial E(k)/\partial k)_\perp|}}
dove  $dk$ rappresenta la quantit\`a di di una piccola orbita nello spazio $k$ e si sta integrando intorno ad un orbita chiusa. Se si vanno a considerare le orbite prime vicine, con differenza di energia $\Delta E = E_2-E_1$ piccola, La regione nel piano $k_x-k_y$ \`e una fascia di larghezza $\Delta(k)$, dove 
\newl{\Delta E = \abs{\left(\frac{\partial E(k)}{\partial k}\right)_\perp } \Delta(k)}
in cui il gradiente dell'energia \`e perpendicolare alla superficie ad energia costante e quindi, la sua proiezione \`e perpendicolare all'orbita. In questo modo il periodo pu\`o essere scritto come
\newl{T=\frac{\hbar^2 c}{\abs{e} H} \frac{1}{\Delta E} \oint\Delta (k) \,dk.}
L'integrale nell'ultima equazione rappresenta l'area tra le due orbite nello spazio $k$, definendo quest'area $\Delta A$, si arriva quindi alla fondamentale relazione
\newl{\boxed{T= \frac{\hbar^2 c}{\abs{e} H} \frac{\Delta A}{\Delta E}},}
in cui si mette in collegamento il periodo orbitale con la geometria della superficie ad energia costante nello spazio $k$. Nel limite della dinamica di elettroni liberi in banda, approssimazione valida per molti metalli, le superfici ad energia costante sono sfere e le orbite sono quindi circolari. Considerando l'energia dell'elettrone libero come $E(k) = \hbar^2 k^2 / 2 m $ l'area delle orbite nel pinao $k_x-k_y$ \`e $A=\pi k^2$, dato che in questo caso ha senso ricondursi al limite semiclassico, dal principio di corrispondenza di Bohr, deduco che
\newl{T=\frac{2\pi}{\omega_C}\,\,\,\,\,\,\,\,\,\,\,\,\,\,\,\, \omega_C = \frac{\abs{e} H}{mc}. \label{CICLOTR:OM}}
In questo modo \`e possibile mettere in relazione la variazione di area trasversale nello spazio $k$ tra orbite adiagenti (tubi di landau adiacenti) e il campo magnetico applicato
\newl{\boxed{\Delta A = \frac{2\pi \abs{e} H}{\hbar c} }.}
La $\omega_C$, definita in (\ref{CICLOTR:OM}), \`e la fondamentale \textbf{frequenza di ciclotrone}. Assomiglia molto alla $\omega$  classica per un elettrone in moto in campo magnetico, rappresentato, per\`o questa volta, nello spazio reale. 
Nella situazione in studio ora, l'energia ha una dipendenza quasi-parabolica dal vettore d'onda $k$, in questo modo le particelle cariche si comportano come particelle libere con frequenza
\newl{\boxed{\omega_C = \frac{\abs{e} H}{m^*c}},}
in cui $m^*$ rappresenta la \textbf{massa effettiva di ciclotrone}. La massa, quindi, pu\`o dipendere dalla direzione del campo magnetico applicato, che \`e collegato alla curvatura della superficie $E(k)$ e quindi in relazione con il tensore di massa efficace della banda.
Per campi magnetici nell'orgine di alcuni $KG$, la tipica frequenza di ciclotrone \`e nelle microonde. Per un campo magnetico di $1T$ e massa e carica dell'elettrone risulta $\omega_C=1.76\cdot10^{11} rad/s$ corrispondente alla lunghezz d'onda di $1.07cm$
\subsection{Livelli energetici di Landau}
Si vuole descrivere ora una trattazione quantistica dei livelli energetici di un sistema di elettroni liberi immersi in un campo magnetico costante. L'energia dei livelli di un sistema di particelle cariche immerse in un campo magnetico uniforme sono ovviamente diversi da quelli di particella libera. Si \`e particolarmente interssati all'energia dei livelli e alla loro distribuzione  nel sistema di elettroni in un metallo. Il primo approccio "geniale" \`e quello dei \textit{Livelli di Landau}.

L'Hamiltoniana del sistema \`e
\newl{\op{H} = \frac{1}{2m}\left(\op{p} -\frac{e}{c}A \right)^2 + e\phi - \mu\cdot H, }
dove $\op{p} $ \`e l'operatore momento dell'elettrone, $A=A(r)$ \`e il potenziale vettore del campo magnetico $H$. Siamo interessati ai livelli energetici di un gas di elettroni liberi in un campo \textbf{magnetostatico}, quindi $\phi=0$. Selezioniamo un campo magnetico diretto solo lungo l'asse $z$ quindi ${\vet H} =H\op{z} $ di modulo $H$. In generale l'operatore momento non commuta con il potenziale vettore infatti si ha che $[\op{p}, A] = -i\hbar\nabla\cdot A$, per\`o si pu\`o scegliere un potenziale vettore, costruito in modo tale che commuti con due componenti di $\op{p} $, che \`e $A = \left(A_x = -H\op{y}   ,A_y=0 ,A_z=0 \right)$, da cui si ottiene facilmente il campo magnetico desiderato (solo lungo l'asse $z$ con modulo $H$). In questo modo l'Hamiltoniana \`e possibile scriverla in modo decisamente pi\`u semplice
\newl{\op{H} = \frac{1}{2m} \left(\op{p} _x + \frac{eH}{c}\op{y} \right)^2 + \frac{\op{p} _y^2}{2m} + \frac{\op{p} _z^2 }{2m} -\mu_z H }
Dato che $[H,s_z]=0$ e dato che il problema \`e tutto concentrato sull'asse $z$ l'ultima parte dell'Hamiltoniana \`e possibile sostituirla direttamente son il suo relativo autovalore
\newl{E_{spin}= \pm \mu_B H = \pm \frac{1}{2} \hbar \omega_C}
cio\`e il contributo energetico di uno spin parallelo o antiparallelo al campo magnetico. Questo contributo energetico diventa rilevante solo se c'\`e un cambiamento nella distribuzione degli spin degli elettroni, come succede nel caso del paramegnetismo di Pauli, dove si hanno valori di suscettivit\`a magnetica piccoli e positivi. 

Escludendo quindi il contributo dello spin, l'equazione di Schroedinger per la funzione d'onda elettronica, nello spazio reale, \`e data da
\newl{\frac{1}{2m}\left[\left(\op{p} _x +\frac{eH}{c}\op{y} \right)^2 + \op{p} _y^2 + \op{p} _z ^2\right]\psi = E\psi. \label{SH:LAND}}
Le compomenti dell'operatore momento, $\op{p} _x$ e $\op{p} _y$ commutano con l'Hamiltoniana e quindi sono costanti del moto. I loro autovalori possono assumere \textit{a priori} valori da $-\infty$ a $+\infty$ senza limitazioni, come per esempio nel modello della particella libera, in cui la funzione d'onda \`e data da onde piane. In particolare, ricordando la relazione di come trasforma il momento in presenza di campi elettromagnetici $m{\vet v} = {\vet p} - (e/c){\vet A} $, si ha che $v_z = p_z/m$ e quindi risulta che la componente $z$ della velocit\`a risulta non quantizzata, mentre le altre componenti della velocit\`a possono essere determinate dagli autovalori dell'energia. Una soluzione per l'equazione (\ref{SH:LAND}) pu\`o essere la seguente ansatz
\newl{\psi(x,y,z) = \exp[i(xp_x+zp_z)/\hbar]\chi(y),}
dove $p_x$ e $p_y$ sono gli autovalori dell'operatore momento. Sostituendo tutto nell'Hamiltoniana, compresa la definizione dell'operatore $p_y=-i\hbar (\partial / \partial y)$, \`e quindi possibile scrivere
\newl{- \frac{\hbar^2}{2m} \frac{d^2\chi}{dy^2} + \left[ \frac{1}{2m}(p_x^2 + p_z^2) + yp_x \frac{eH}{mc} + y^2 \frac{e^2H^2}{2mc^2} \right]\chi = E \chi.  }
Introducendo la frequenza di ciclotrone ed introducendo il parametro $y_0 = -cp_x/(eH)$ si ottiene in conclusione il seguente problema agli autovalori
\newl{\left[ -\frac{\hbar^2}{2m} \frac{d^2}{dy^2} + \frac{m}{2}\omega_C^2(y-y_0)^2 \right]\chi(y) = \left(E-\frac{p_z^2}{2m}\right)\chi(y),}
che \`e praticamente l'equazione di Schroedinger per un oscillatore armonico monodimensionale, con $\omega = \omega_C$, centrato in $y_0$. La funzione d'onda e gli autovalori dell'energia, a questo punto, sono ben noti
\newl{E_n - \frac{p_z^2}{2m} = \hbar\omega_C\left(n+\frac{1}{2}\right)}
dove $n$ \`e intero e rappresenta il numero quantico riferito al corrispondente \textbf{Livello di Landau}, per un elettrone in moto in un campo magnetico uniforme. Riscrivendo meglio la relazione precedente
\newl\boxed{{E_{n_L}(p_z) = \hbar\omega_C\left( n_L + \frac{1}{2}  \right) + \frac{p_z^2}{2m}.}}





