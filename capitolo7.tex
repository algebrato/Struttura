\section{Magnetismo nella Materia}
L'obiettivo del seguente capitolo è quello di introdurre una serie di proprietà di origine magnetica, quindi parleremo di \textit{Magnetismo nella materia}. A differenza degli aspetti macroscopici studiabili con i mezzi dell'elettrodinamica classica, ci si concentrerà sugli aspetti magnetici microscopici della materia e il loro collegamento col le variabili termodinamiche. Lo studio della materia sotto questo punto di vista ha permesso lo sviluppo di concetti quali: \textit{ordinamento spontaneo}, \textit{transione di fase} e \textit{approssimazione di campo medio}.
Partendo inizialmente dallo studio di paramagnetismo e diamagnetismo sarà possibile introdurre il concetto di \textit{approssimazione di particelle indipendneti} e successivamente passare allo studio delle \textit{correlazioni tra particelle in siti differenti} e quindi avere i mezzi di interpretazione del ferromagnetismo nella materia. Un possibile punto di partenza del discorso è sicuramente il teorema di Bohr - Van Leeuwen che afferma che la magnetizzazione nella materia ${\vet M}$ è sempre nulla. Nonostante questo sono evidenti i dipoli così come sono evidenti i fenomeni di ordinamento magnetico. Si inizi col definire la nomenclatura che verrà usata, per esempio tutto verrà scritto in unità c.g.s. In questo modo il campo di induzione magnetica è definito come
\newl{\boxed{{\vet B} = {\vet H} + 4\pi{\vet M}}}
dove con ${\vet M}$ si vuole identificare il vettore magnetizzazione che non è altro che il vettore di momento di dipolo di ogni singolo atomo, mediato sul volume, quindi una cosa del tipo
\newl{{\vet M} = \frac{{\vet \mu}}{V}.} 
Definisco la suscettività magnetica $\chi_M$ come 
\newl{\chi_M = \left(\frac{\partial M}{\partial H}\right)_{H=0}}
considerando in modo approssimato una relazione lineare tra $\vet M$ ed $\vet H$ e possibile scrivere che $M = \chi_M H$. \`E possibile riconoscere, in funzione al valore di $\chi$ diversi comportamenti della materia:
\begin{itemize}
	\item[1.] \textbf{Diamagnetismo}: caratterizzato da $\chi<0$, il sistema tende a "rifiutare" il campo applicato. Esempio: precessioni di Larmor;
	\item[2.] \textbf{Paramagnetismo}: caratterizzato da $\chi>0$, in questo caso se gli  atomi hanno un loro momento magnetico intrinseco, tendono ad allinearlo in direzione del campo;
	\item[3.] \textbf{Ferromagnetismo}: caratterizzato da fenomeni particolari quali \textit{magnetizzazione spontanea} e \textit{ordinamento}. I dipoli magnetici tendono ad orientarsi spontaneamente. In qeusto caso non vale più l'approssimazione lineare $M = \chi_M H$, in quanto il sistema sarà caratterizzato da transizioni di fase e in generale, da comportamenti abbastanza peculiari.
\end{itemize}
C'è un legame molto forte tra questi fenomeni e la temeperatura, quindi è opportuno fare una trattazione termidinamica. Si parte dall'enunciazione del primo principio:
\newl{dU = dQ - dW.}
Il lavoro per un genereico sistema immerso in un campo magnetico sarà dato, appunto, in parte da lavoro magnetico, una perte di lavoro meccanico e una parte di assorbimento dovuta all'eventuale scambio di particelle, indicando con $\zeta$ il potenziale chimico, 
\newl{dW = \mu\cdot dH + PdV - \zeta dN}
\begin{figure}
	\usetikzlibrary{arrows}
	\centering
	\fbox{
		\begin{tikzpicture}[scale=1,auto=center]
			\newcommand{\midarrow}{\tikz \draw[-triangle 90] (0,0) -- +(.1,0);}
			\draw[->] (0,-1) -- (0,5);
			\draw[->] (-1,0) -- (5,0);
			\node[] at (-0.2,-0.2) {$O$};
			\node[fill,thick,circle, inner sep=0pt, minimum size=0.2cm] at (3,3)  {};
			\draw (1,1) -- node {\midarrow} (2,1);
		\end{tikzpicture}
	}
	\caption{Forze in gioco}
	\label{Lorenz}
\end{figure}
Differenziando l'espressione dell'energia libera si ottiene che
\newl{dF = dU -d(TS) = -SdT -PdV +\zeta dN -\mu\cdot dH} 
e si ottengono tutte le relative grandezze in funzione di $F$ a $T$ $V$ ed $N$ fissati,
\newl{M = -\frac{1}{V}\left(\frac{\partial F}{\partial H}\right)_{T,V,N}\,\,\,\,\,,\,\,\,\,\,\chi = -\frac{1}{V}\left(\frac{\partial^2 F}{\partial H ^2}\right)_{T,V,N} .}
In meccanica statistica la funzione di partizione posso ottenerla anche dalla funzione di partizione canonica, che somma su tutti gli stati microscopici[\textbf{frase senza senso!!!}]
\newl{F = -k_BT\log Z \,\,\,\,,\,\,\,\, Z= \sum_{stati} e^{-H/k_BT}}
dove $k_B$ è la costante di Boltzmann e H è la classica Hamiltoniana di sistema. In alcuni casi è comodo scegliere il potenziale chimico come una rilevante variabile termodinamica, da cui si ottiene il potenziale \textbf{Gran Canonico}
\newl{\Omega = F -\zeta N = \Omega(T,V,\zeta,H)} 
a cui è associata una funzione di partizione gran canonica
\newl{Z_{\Omega}= \sum_{stati,particelle} e^{(\zeta N -H)/k_BT}}
dove la somma è appunto condotta su tutti gli stati e su tutte le particelle del sistema. Differenziando ottengo
\newl{d\Omega = dF -d(\zeta N) = -SdT -PdV -Nd\zeta - \mu\cdot dH}
da cui si derivano ancora le varibili osservabili del sistema
\newl{M = -\frac{1}{V}\left(\frac{\partial \Omega}{\partial H}\right)_{T,V,\zeta} \,\,\,,\,\,\, N =-\left(\frac{\partial \Omega}{\partial \mu}\right)_{T,V,H}}
Differenziando l'energia libera in funzione al gran potenziale canonico
\newl{\left(\frac{\partial F}{\partial H}\right)_N = \left(\frac{\partial(\Omega+\zeta N)}{\partial H}\right) = \left(\frac{\partial \Omega}{\partial H}\right)_\zeta + \frac{\partial \Omega}{\partial \zeta}\frac{\partial \zeta}{\partial H} + N \frac{\partial \mu}{\partial H}}
Come visto precedentemente la magnetizzazione è definita come una funzione di $\zeta$, quindi per ottenere la suscettività magnetica in funzione al numero di particelle si deve sempre partire dalla scrittura del gran potenziale canonico, rimaneggiarlo, ottenendo che
\newl{\chi = \left(\frac{\partial M(\zeta_{H,N}, H)}{\partial H}\right)_N = \frac{\partial M(\zeta,H)}{\partial H} + \frac{\partial M(\zeta,H)}{\partial \zeta}\frac{\partial \zeta}{\partial H}.}
Nella maggior parte dei casi, il potenziale chimico ha una dipendenza quadratica dal modulo del campo magnetico, quindi possiamo considerare $(\partial \zeta / \partial H) \sim 0$ e la suscettività data direttamente dalla formula
\newl{\chi = \left(\frac{\partial M}{\partial H}\right)_{H=0} = -\frac{1}{V} \left(\frac{\partial^2\Omega}{\partial H^2}\right)_\zeta.}
\subsection{Teorema di Bohr - Van Leeuwen}
Viene ripresa la tesi del teorema BVL e ampliata nei concetti. Come detto precedentemente, dal punto di vista della meccanica statistica classica, non sono ponderati i fenomeni magnetici nella materia, indipendenetemente da T e dall' {\vet H} applicato. Questo lo si può mostrare in modo semplice prendendo il valore medio di un'osservabile generica $F(r_i,p_i)$
\newl{\langle F \rangle = \frac{\int d^3r_i d^3p_i F(r_i,p_i) e^{-H(r_i,p_i)/k_BT}}{\left[\int d^3r_i d^3p_i e^{H(r_i,p_i)/k_BT}\right]} }
in presenza di campo magnetico, è risaputo che $p_i$  trasforma attraverso trasformazioni minimali $p_i \to p_i -(e/c)A(r_i) = p'_i$ mentre $r_i$ non cambia. Riscrivendo l'integrale di prima, la parte di potenziale vettore sparisce, perchè dipende solo dagli $r_i$ quindi ottengo che:
\newl{\langle F \rangle_{\text{senza A}}=\langle F \rangle_{\text{con A}}}
da cui si può dedurre che l'eventuale valore medio dell'osservabile $M$ con $H=0$, è uguale a quello per $H\neq0$ perchè il contributo del campo magnetico sparisce. Quindi ho che $\forall H$ 
la magnetizzazione di un sistema è sempre zero. Affermazione totalmente falsa, l'evidenza macroscopica del ferromagnetismo e delle transizioni di fase sono banali. Si conclude che un medello di meccanica statistica classica, non dice nulla riguardo i fenomeni magnetici. I fenomeni di ordinamento spontaneo sono fenomeni miscroscopici che vanno trattati in modo quantistico, infatti per prima cosa $\left[p,A\right]\neq 0$ ed in più va introdotto il concetto di dipolo magnetico intrinseco puramente quantistico, legato allo spin
\newl{\op{\mu_s} = g_s\frac{e}{2mc}\op{S} = -2\mu_B \op{S} \,\,\,,\,\,\,\op{S} _z=\pm\frac{\hbar}{2} }
\subsection{Hamiltoniana di Interazione Magnetica}
Come detto nel precedente paragrafo, il ferromagnetismo ed i fenomeni di ordinamento scpontaneo sono puramente quantistici, dunque il primo passo consiste nello scrivere l'Hamiltoniana del sistema in studio. Per un sistema di un elettrone immerso in un campo magnetico, si ha 
\newl{\op{H} = \frac{1}{2m}\left(\op{p} -\frac{e}{c}A(\op{r} )\right)^2.}
Per avere invarianza dei campi uso una gauge di Coulomb quindi con $\nabla\cdot{\vet A} =0$, che mi permette di scrivere il potenziale vettore e il campo come 
\newl{{\vet A} = \frac{1}{2} {\vet H} \times \vet{r}\,\,\, \implicaa\,\,\, {\vet H = \bf{\nabla} \times {\vet A} }}
in più, nella gauge di Coulomb ho che $\left[p,A\right]=0$. Riscrivendo l'Hamiltoniana di partenza diventa
\newl{\op{H} = \frac{\op{p} ^2}{2m} + \frac{e^2}{8mc^2}({\vet H} \times {\vet r})^2 - \frac{e}{2mc} \op{p} \cdot ({\vet H} \times {\vet r}) }
Supponendo che il campo magnetico sia diretto solo lungo l'asse $\op{z} $ e usando le seguenti relazioni vettoriali
\newl{
	({\vet H}\times {\vet r})^2 &=& H^2r^2 -({\vet H}\times {\vet r})^2 = H^2(x^2+y^2) \nonumber\\
	p\cdot({\vet H}\times {\vet r}) &=& {\vet H}\cdot {\vet r}\times {\vet p} = {\vet H}\cdot {\vet L} = H_z L_z
}
la parte cinetica dell'Hamiltoniana diventa 
\newl{H_c = \frac{p^2}{2m} + \frac{e^2}{8mc^2} H^2(x^2+y^2) -\frac{e}{2mc}L_zH}
il termine quadrato è dovuto al moto nel piano $x-y$, il termine lineare ha la forma di una energia di interazione tra il campo $\vet H $ e il momento di dipolo magnetico $\mu_l$ associato all'elettrone in orbita. In questo modo è ben distinguibile il comportamento paramagnetico, infatti considerando $l=L/\hbar$ ottengo
\newl{\mu_l = -\frac{\abs{e} \hbar }{2m}l = -\mu_B l}
che permette di scrivere l'Hamiltoniana di interazione 
\newl{H_{int} = -\mu_l H = +\mu_B l_z H = -\frac{e}{2mc}L_z H}
inserendo anche i contributi di interazione con campo elettrico, interazione del campo con lo spin e l'interazione spin-orbita ottengo l'Hamiltoniana totale di un sistema immerso in un campo elettromagnetico, definendo $\omega_c = (eH)/(mc)$ la omega di ciclotrone si ottiene
\newl{\op{H} = \frac{\op{p} ^2 }{2m}  + e\varphi(r)  + \frac{m}{8}\omega_c(x^2-y^2) + \mu_B l_z H + 2\mu_B S_z\cdot H +\xi(r) \op{L} \cdot \op{S}   }
le parti elettostatiche e di spin-orbita sono state aggiunte perchè l'elettrone è all'inteno di un atomo, quindi è soggetto anche a questo tipo di interazioni. \`E possibile ora discutere sui termini presente nell'Hamiltonianao. Il primo termine è quello cinetico, il secondo è quello di interazione con il potenziale elttrostatico, il terzo rappresenta lo sviluppo al prim'ordine dell'interazione col campo magnetico, il quarto rappresenta lo sviluppo al second'ordine sempre dell'interazione col campo magnetico, in quinto rappresenta l'interazione dello spin col campo magnetico ed infine si ha l'interazione spin-orbita. \`E possibile notare che \newl{\left[\frac{m}{8}\omega_c^2 (x^2-y^2) \right]  > 0}
quindi è un contributo repulsivo dimagnetico. Dal punto di vista delle grandezze fisiche, però ha un contributo di $\sim 10^{-10}eV$ che molto più piccolo del contributo dei termini quattro e cinque, che si aggirano sui $\sim 10^{-4}eV$ che rappresentano il contributo paramagnetico. Anche lo spin-orbita è di questo ordine di grandezza, $\sim 10^{-2}eV \to 10^{-4}eV$.
Poichè le energie dei livelli elettronici sono nell'ordine dell'$eV$ per eccitare un elettrone sarebbe necessario un campo magnetico gigantesco. In sostanza non è possibile eccitare gli elettroni tramite campi magnetici, quindi tutti gli effetti che sono legati all'interazione magnetica avvengono tutti nel ground state.
\subsection{Materiali Isolanti - Legge di Curie}
I materiali isolanti sono caratterizzati da atomi le cui bande sono piene. I numeri quantici utili per studiare questo tipo di sistemi sono $L,S,J$ totali, cioè $L=\sum_i l_i$ ecc... Dato lo stato $\psi_{r,s,j,j_z}$ il contributo paramagnetico è dato da
\newl{ \bra{J,J_z} L_z+2S_z \ket{J,J_z} = \mu_B Hg_LJ_z }
che è l'effetto Zeeman. Quindi per questi materiali la parte di Hamiltoniana di interazione sarà
\newl{\op{H} _{par} = \mu_B Hg_L\op{J} _z .}
Di notevole interesse è notare il fatto che l'intesità della correzione all'energia col contributo paramagnetico è nell'ordine di $10^4eV <k_BT$ ! Quindi sul sistema entrano in gioco anche le fluttuazioni dovute alla temperatura, che essendo dello stesso ordine di grandezza, non possono essere trascurate. \`E necessario quindi effettuare una trattazione del problema utilizzando i metodi della meccanica statistica. Come notato, si ha che $k_BT\sim10^{-2}eV \gg E_M$ quindi è necessaria una trattazione statistica. Il primo passo è quello del calcolo della funzione di partizione
\newl{Z=\sum_{stati}e^{-\frac{H}{k_BT}} = \sum_{-J}^{+J} e^{-\frac{\mu_B H g_L J_z}{k_BT}}}
ponendo $\gamma = (g_L \mu_B H) / (k_B T)$ e $J_z = n-J$ la funzione di partizione diventa
\newl{Z = e^{\gamma J} \sum_{n=0}^{2J}e^{-\gamma n} } 
che è molto facile sommare in quanto è una geometrica 
\newl{Z = e^{\gamma J} \frac{1-e^{-\gamma(2J+1)}}{1-e^{-\gamma}} = \frac{\sinh[\gamma(J+1/2)]}{\sinh(\gamma/2)}}
avendo a disposizione la funzione di partizione è possibile quindi passare alla determinazione di tutte le altre variabili termodinamiche del sistema. Di particolare interesse in questo capitolo è la determinazione della magnetizzazione
\newl{M = -\frac{N}{V}\frac{\partial}{\partial H}\left(-k_B T \ln Z\right) = \frac{Nk_BT}{V} \frac{1}{Z} \frac{\partial Z}{\partial \gamma} \frac{\partial \gamma}{\partial H}}
facendo i conti si arriva all'espressione della magnetizzazione in funzione alla funzione di Brillouin
\newl{M = \frac{N}{V}g_L \mu_B J \mathcal{B}_J\left(\frac{g_L\mu_BJH}{k_BT}\right)}
dove la funzione di Brillouin è una funzione adimensionale data da
\newl{\mathcal{B}_J = \frac{2J+1}{2J} \coth\left(x\frac{2J+1}{2J}\right)- \frac{1}{2J}\coth\left(\frac{x}{2J}\right)}
a temperatura ambiente l'energia magnetica è sempre più piccola dell'energia termica $(\mu_B H \ll k_BT \to x\ll 1)$ quindi è possibile espandere la $\coth(x)$ come  
\newl{\coth(x) = \frac{1}{x} + \frac{x}{3} + \cdots}
in questo modo si ottiene una espressione lineare per $M$
\newl{M = \frac{N}{V} \frac{g_L^2 \mu_B^2 J(J+1)}{3k_BT} H}
Infine si raggiunge l'espressione della suscettività paramagnetica di \textbf{Curie} in cui è visibile la dipendenza dall'inverso della temperatura
\newl{\chi_{para} = \frac{N}{V} \frac{(g_L \mu_B)^2 J(J+1)}{3k_B T}.}
Questo suggerisce che l'atomo può essere trattato come se avesse un suo proprio momento magnetico classico. La legge funziona molto bene per quegli ossidi che hanno la shell $f$ semipiena, mentre per le transizioni nei metalli funziona solo se sono verificate le condizioni $L=0$ e $J=S$. Per temperature molto basse e campi magnetici molto intensi il limite si inverte e si ha
\newl{x\to1 \,\,\,\, \coth(x)\to 1,}
e allo stesso modo si ha che la funzione di Brillouin tende ad $1$, $\mathcal{B}_J \sim 1$ e la magnetizzazione non dipende più da $H$ e si ottiene una magnetizzazione di saturazione
\newl{M=\frac{N}{V}g_L \mu_B J .}
Questa formula si interpreta facilmente dicendo che si raggiunge una situzione di saturazione, in cui tutti i momenti di dipolo di intensit\`a $g_L \mu_B J$ sono perfettamente allineati col campo $H$. Per alte temperature l'energia termica \`e molto pi\`u grande e i dipoli sono diretti equamente in tutte le possibili direzioni. Una stima numerica della magnetizzazione di saturazione, per una data densit\`a, che possiamo fissare a $N/V\sim 10^{22} atom/cm^{3}$, si aggira circa su $M\sim10^2 G = 10^{-2}T$. In pratica questo valore non viene mai raggiunto nei mezzi paramagnetici a temperatura ambiente, ma solo nei materiale ferromagnetici, infatti il paramagnetismo [\`e il piu grande effetto magnetico][ma che stai a d\`i?], ma la sua magnetizzazione e la sua suscettivit\`a magnetica hanno valori numerici molto piccoli, per esempio, per un campo di $H=1T$ si ottiene una $M\sim 1G$ e una $\chi \sim 10^{-4}$, giustificando l'assunsione che $B\sim H$.

Per quanto riguarda la scarsa valit\`a della legge di Curie per gli elementi di transizione della tavola periodica, in particolare gli atomi con una shell $d$ incompleta, questi atomi sono caratterizzare dall'avere orbitali molto estesi spazialmente anche oltre al raggio atomico. Questa particolarit\`a in una struttura cristallina potrebbe portare gli atomi ad interagire, via interazione Coulombiana con altri elettroni o ioni. Questa interazione potrebbe essere pi\`u intensa dell'interazione Spin-Orbita, in questo modo l'originale degenerazione sull'orbitale $d$ viene rotta.
Solitamente si considera una combinazione lineare di orbitali , corrispondenti gli stati nuovi con energie diverse (Crystal Field Splitting).In queste situazioni il momento angolare $L$ compie precessioni complicate all'interno del campo cristallino e la sua proiezione $L_z$ non \`e una costante del moto del sistema, quindi non \`e per nulla un buon numero quantico. Il valore di aspettazione delle proiezioni di $L$, tuttavia, sono $\langle L_x \rangle = \langle L_y \rangle = \langle L_z \rangle = 0$, risultato conosciuto come \textbf{Quenching del momento angolare orbitale}.
In pratica, per questo tipo di atomi in stato \textit{condensato} hanno $L=0$ e la legge di Curie \`e valida. Differente \`e il caso di per atomi in cui \`e presente la shell $f$ completa, ma gli stessi orbitali $f$ sono ben confinati spazialmente all'interno dell'atomo. I primi $4f$ orbitali iniziano ad essere riempiti dagli elettroni dopo il completamento dell'orbbitale $6s$ che \`e MOLTO ESTERNO. il Crystal field splitting \`e piccolo rispetto all'interazione all'accoppiamneto spin-orbita e le propriet\`a magnetiche dell'atomo sono molto simili a quelle di un sistema isolato. La legge di Curie rimane comunque valida in quanto il campo cristallino potrebbe originare piccole anisotropie magnetiche che conferiscono ai dipoli magnetici un orientamento preferenziale.
Casi speciali sono rappresentati dai cos\`i detti elementi di transizione. Possiamo identificare due diversi tipi:
\begin{itemize}
	\item Elementi con $J=0$ e $L = S \neq 0$;
	\item Elementi con $J = L = S = 0$.
\end{itemize}
\subsection{Elementi con $J=0$ e $L = S \neq 0$}
Questa situazione si verifica quando l'ultima shell ha un elettrone in meno per essere completata a met\`a. Un esempio diretto di questo fatto \`e lo ione di Europio $\text{Eu}^{3+}$ a bassa temperatura. Il contributo alla magnetizzazione al prim'ordine, usando l'Hamiltoniana effettiva \`e esattamente uguale a zero, quindi ci si appella al second'ordine perturbativo facendo diventare il contributo energetico
\newl{ \Delta E = - \sum_{n\neq0}\frac{ \left|  \bra{0} \mu_BH(L_z+2S_z) \ket{n} \right| ^2   }{E_n - E_0} }
dove $n$ rappresenta la collezione di numeri quantici particolarmente adatti dello stato eccitato del sistema. $\bra{0} $ rappresente il ground state e $\ket{n} $ rappresenta la funzione d'onda associata allo stato imperturbato di energia $E_n$. Da questa variazione di energia, proporzionale ad $H^2$ \`e possibile mettere in luce piccolissime suscettivit\`a paramagnetiche, praticamente indipendenti dalla temperatura. Questo fenomeno \`e conosciuto come \textbf{paramagnetismo di Van Vleck}. Queste oscillazioni sono molto difficili da osservare in quanto sono dello stesso ordine del contributo diamagnetico, che solitamente li maschera.
\subsection{Elementi con $J = L = S = 0$}
Sono tutti gli atomi con shell esterna completa. Questo \`e il caso in cui si studia il diamagnetismo. Il contributo magnetico all'energia \`e rappresentato solo dalla parte quadratica dell'Hamiltoniana effettiva. Il valore di aspettazione sul ground state \`e
\newl{\Delta E = \frac{m}{8} \omega^2_C  \bra{0} \sum_i(x_i^2 + y_i^2) \ket{0}    }
dove la somma \`e svolta su tutti gli $Z$ elettroni dell'atomo. Sfruttando la simmetria sferica del sistema, $x_i^2 = y_i^2 = z_i^2 = r_i^2 / 3$  \`e possibile riscrivere il contributo magnetico all'energia come 
\newl{\Delta E = \frac{e^2H^2}{12 m c^2} \bra{0} \sum_i r_i^2 \ket{0}  ,}
e la suscettivit\`a magnetica risulta
\newl{\chi_{dia} = -\frac{N}{V} \frac{e^2}{6mc^2} \bra{0} \sum_i r_i^2 \ket{0}  .}
Questo tipo di diamagnetismo \`e visibile nei gas nobili ed \`e l'esatto analogo classico del diamagnetismo di Larmor, dovuto al moto orbitale di particelle cariche. 
\subsection{Riassunto sugli elementi di transizione}
Riassumendo, se $J=0$,si ha che si annulla tutto, quindi devo passare al secondo ordine perturbativo attraverso la teoria di \textbf{Van Vleck}. Se $J=L=S=0$ allora tutti gli oggetti hanno suscettività diamagnetica. Sperimentalmente si osserva che per ioni con shell $f$ semipiena e $J\neq0$ l'accordo con la teoria è molto buono. Per gli elementi di transizione (shell $d$) la teoria non funziona per nulla. Per i casi in cui $J=0$ i il primo ordine perturbativo è nullo [\textbf{Non \`e vero !! Verificare....}]
\newl{\bra{\psi} \mu_BH(L_z+2S_z) \ket{\psi} = \mu_BHg_s J_z =0,}
devo usare il secondo ordine pertubativo rappresentato dalle oscillazioni di Van Vleck.


\subsection{Gas di Elettroni Liberi - Paramagnetismo di Pauli}
Nei metalli, i fenomeni di interazione con il campo elettromangetico sono dovuti per lo più dalla risposta degli elettroni liberi nelle bande di conduzione, al campo applicato. Gli effetti di magnetizzazione nei metalli sono caratterizzati da due coportamenti indipendenti:
\begin{itemize}
	\item Paramagnetismo di Pauli $\to$ dovuto all'allineamento dei dipoli magnetici col campo $\vet{H}$;
	\item Diamagnetismo di Landau $\to$ il secondo dovuto alla quantizzazione del moto degli elettroni immersi in un campo magnetico;
\end{itemize}
Gli elettroni nelle bande dei metalli hanno notevoli gradi di libertà quindi l'idea di trattare gli elettroni come un gas è ragionevole e d\`a risultati consistenti. Si calcolerà la suscettività paramegnetica di Pauli per un gas di elettroni. La derivazione può avvenire per vai statistica o per via variazionale, qui di seguito sarà argomentata la via statistica. Consideriamo l'ambientazione del problema in un sistema \textbf{gran canonico}. L'Hamiltioniana di riferimento \`e 
\newl{H=\frac{1}{2m^*} \left({\vet p} - \frac{e}{c}{\vet A} \right)^2 + 2\mu_BS_zH  }
con la parte cinetica data dall'interazione col campo magnetico e la seconda parte che rappresenta il termine paramegnetico di Pauli. La funzione di partizione, per particelle fermioniche, sarà quella gran canonica data da
\newl{Z_\Omega = \prod_{k_x,k_y,k_z,s_z} \left(1+e^{(\mu-E(k,s_z))/k_BT}\right). }
La produttoria è svolta su tutti i singoli stati degli elettroni con vettor d'onda 
$\vet{k} = \left(k_x,k_y,k_z\right)$ e spin $s_z$ aventi energia $E(\vet{k} ,s_z)$.
In assenza di campo magnetico, l'energia \`e quella di particella libera $E_0(k) = \hbar^2k^2/2m (k^2=k_x^2+k_y^2+k_z^2)$ per ogni proiezione dello spin lungo l'asse $z$. All'equilibrio termico met\`a degli spin saranno up e met\`a down. La funzione di partizione sar\`a $Z_0$, il potenziale gran canonico sar\`a $\Omega_0(\mu)$ dove $\mu$ rappresenta un adeguato potenziale chimico.
In presenza di campo magnetico $H$, l'energia si modificher\`a avendo in pi\`u il termine di interazione magnetica $-{\vet \mu_s} \cdot {\vet H} $, in particolare (all'equilibrio termico) gli elettroni con spin up, daranno un contributo all'energia pari $+\mu_BH$ mentre quelli con spin down $-\mu_BH$. L'energia degli stati sar\`a quindi 
\newl{E_{up/down} = E_0 \pm \mu_B H}
La funzione di partizione gran canonico ora dipender\`a dal campo magnetico. Pu\`o essere scritta separando le due famiglie di stati presenti
\newl{Z_{\Omega}(H) = \prod_{k_{up}} \left(1+e^{(\mu-\mu_BH-E_0(k))/k_BT} \right)\cdot \prod_{k_{down}} \left(1+e^{(\mu+\mu_BH-E_0(k))/k_BT} \right).}
Dalla differenza tra $Z_{\Omega} (H)$ e $Z_0$ la funzione di partizione gran canonica, alla fine, risulta dipendente solo dai due stati
\newl{\Omega(\mu,H) = \frac{1}{2}\Omega_0(\mu-\mu_BH) + \frac{1}{2}\Omega_0(\mu+\mu_BH).}
Il termine $\mu_BH \ll \mu$ (negli ordine di parecchi eV), quindi possiamo espandere la formula precedente in $\mu_BH$ ottenendo
\newl{\Omega(\mu,H) \sim \Omega_0(\mu) + \frac{1}{2}\mu^2_BH^2 \left.\frac{\partial^2\Omega}{\partial\mu^2}\right|_{H=0} }
dato che $N = - (\partial\Omega / \partial\mu)$ possiamo usarla nella formula della suscettivit\`a 
\newl{\chi_{pauli}= -\frac{1}{V} \frac{\partial^2\Omega}{\partial H^2} = \frac{1}{V} \mu_B^2 \left(\frac{\partial N}{\partial \mu}\right)_{T,V}  }
Per un sistema di fermioni posso facilmente calcolare $N$ come 
\newl{N = \int_0^{+\infty}dE\,g_{3D}(E)\left(1+e^{\frac{E-\mu}{k_BT}}\right)^{-1} = \int_0^{E_F}dE\,g_{3D}(E) = \frac{2}{3}\frac{V}{2\pi^2}\left(\frac{2m}{\hbar^2}\right)^{\frac{5}{2}}E_F^{\frac{3}{2}}}
in questo modo \`e possibile scrivere la sucettivit\`a di Pauli come
\newl{\chi_{pauli}=\frac{3}{2}\frac{N}{V}\frac{\mu_B^2}{E_F}  = \frac{3}{2}\frac{N}{V}\frac{\mu_B^2}{k_BT} .}
\textbf{N.B.} = $\chi_{pauli} \sim \chi_{para}/100$, questo \`e legato al principio di indeterminazione di Pauli. Considerando le sfere di fermi, solo gli elettroni vicini alle superfici possono riallineare lo spin, quelli interni non possono perch\'e gli stati con spin "dritto" sono gi\`a occupati. Solo una piccola parte di elettroni puo' cambiare segno. Al limite di $T \gg T_F$ si ritorna alla legge di Curie. [Capire bene perch\`e...]. 
\subsection{Gas di Elettroni Liberi - Diamagnetismo di Landau}
In questa sezione studieremo gli effetti che un campo mangetico ha sul moto libero degli elettroni nella banda di conduzione, indipendentemente dal loro spin. Fin'ora \`e stato visto che per il limite di temperatura molto bassa ($k_BT\ll\hbar\omega_C$) il tutto veniva spiegato con l'effetto di De Haas - Van Alphen. Ora verr\`a tolta questa condizione e si attaccher\`a il problema del magnetismo dell'orbitale elettronico per TUTTE le temperature, usando sempre i metodi della termodinamica e della meccanica statistica. I livelli energetici del sistema vengono dati dalla teoria di Landau, e sono quelli di un oscillatore armonico bidimensionale di frequenza angolare $\omega_C$ pi\`u un contributo di particella libera lungo l'asse $z$
\newl{E_n(k_z) = \frac{\hbar^2k_z^2}{2m} + \hbar\omega_C\left(n+\frac{1}{2}\right), }
mentre ogni levello ha una degenerazione data da 
\newl{N_d = \frac{|e|AH}{\pi\hbar c} = \frac{mA}{\pi \hbar^2}\cdot \hbar\omega_C}
dove $A$ rappresenza la superficie del campione [trasversale?? a cosa??]. La funzione di gran partizione diventa quindi
\newl{Z_\Omega = \prod_{k_x,k_y,k_z,s_z}\left[1+e^{\frac{\mu-E_n(k_z)}{k_BT}}\right].}
Per semplicit\`a introduciamo il parametro $\mu_z$ definito come 
\newl{\mu_z =  \mu - \frac{\hbar^2k_z^2}{2m}} 
e riscriviamo il gran potenziale come $\Omega = -k_BT\ln Z_\Omega = \sum_{k_z} \Omega'_z$ nella forma di una somma su di un due-dimensionale potenziale definito per un certo valore fisso di $k_z$ dove i vari $\Omega'_z$ sono definiti come 
\newl{\Omega'_z &=& -k_BT \sum_{k_x, k_y, s_z} \ln\left[1+e^{\frac{\mu_z-E_n}{k_BT}}\right] = \nonumber  \\
		&=& -N_dk_BT\sum_{n=0}^{+\infty} \ln\left[1+e^{\frac{\mu_z-E_n}{k_BT}}\right],}
dove $E_n=\hbar\omega_C(n+1/2)$ ed \`e stata introdotta la degenerazione degli stati perch\`e questo numero era stato esattamente calcolato come numero totale di stati elettronici permessi sulla superficie piana $A$ ad un fissato $n$. La dipendenza dal potenziale bidimensionale da $H$ \`e introdotta dalla frequenza di ciclotrone $\omega_C$ presente in $E_n$ e anche dal potenziale chimico, in cui, in realt\`a sarebbe un $\mu=\mu(H)$. Nella formula sopra la somma sulla tripla $\{k_x, k_y, s_z\}$ pu\`o essere pensata come somma su tutti i livelli energetici e su tutte le degenerazioni. Le degenerazioni le conosco dalla teoria di Landau. La somma scritta sopra si potr\`a notare che \`e possibile scriverla come integrali di termini oscillanti. Per ottenere dei risultati \`e necessario riscrivere la somma infinita come un integrale facile da calcolare. L'appossimazione che viene fatta \`e quella di ussare uno sviluppo in serie. La parte di particella libera non dipende dal campo H, la prima parte \`e la funzione di gran partizione per la particella libera in assenza di campo magnetico, solo la parte di oscillatore armonico conetiene una dipendenza da H quindi la parte di gran potenziale dovuta alla presenza di campo magnetico diventa
\newl{\Delta\Omega(H) = \sum_{k_z} \frac{mA\omega_C^2}{24\pi} \frac{1}{1+e^{\frac{-\mu_z}{k_BT}}}.}
In questo modo modo \`e possibile scrivere la funzione di partizione gran canonica e arrivare a calcolare la suscettivit\`a magnetica che risulta essere
\newl{\chi_L = -\frac{e^2}{12m\pi^2c^2}k_F = -\frac{1}{3}\frac{g_{3D}(E_F)}{V}\mu_B^2 = -\frac{1}{2} \frac{N}{V} \frac{\mu_B^2}{E_F}}
SI OSSERVI che la formula per la suscettivit\`a diamagnetica di Landau, ricorda molto quella paramagnetica di Pauli, in particolare il legame tra le due scritture \`e 
\newl{\boxed{\chi_L = -\frac{1}{3} \chi_P }}
dunque entrambi gli effetti concorrono con lo stesso ordine di grandezza alla risposta magnetica del gas di elettroni liberi che, in conclusione, \`e il modello pi\`u semplice per rappresentare la risposta magnetica dei metalli.

In questi sistemi il relativo peso del diamagnetismo contro il comportamento paramagnetico \`e fisso ad un $1/3$, ma esistono sistemi con un rapporto differente tra le due scuscettivit\`a. Si consideri un semiconduttore dopato (DS), in questo caso \`e presente un relativo minor numero di elettroni (o buche) nelle bande o vicini al comportamento di particella libera con massa effettiva $m^*$. La dipendenza della suscettivit\`a dalla massa segue dal fatto che nella $\chi_P$ \`e presente la densit\`a degli stati che $\partial N/\partial\mu=g_{3D}(E_F)\propto m$, in questo modo la suscettivit\`a paramegnetica scala come $\chi_P^{DS} = (m^*/m)\chi_P^{free}$.

Al contrario, nella $\chi_L$ la massa entra attraverso i livelli energetici di Landau che sono definiti da $E_n = \hbar\omega_C\left(n+\frac{1}{2}\right) \propto 1/m$, infatti il gran potenziale e $\chi_L$, sono inversamente proporzionali a $m$. Da queste osservazioni, nel caso di semiconduttori dopati, si ha che $\chi_L^{DS} = (m/m^*) \chi_L^{free}$.

In conclusione, per semiconduttori dopati, il rapporto tra le due suscettivit\`a risulta essere
\newl{\boxed{\frac{|\chi_L^{DS}|}{\chi_P^{DS}} = \frac{1}{3}\left(\frac{m}{m^*}\right)^2 ,}}
dato che tipicamente $m^* \sim 0.1m$ si ha che il comportamento diamagnetico diventa dominante rispetto a quello paramagnetico, e pu\`o essere direttamente misurato comparando il campione dopato con una non dopato.
\subsection{Fenomeni di Ordinamento Magnetico - Ferromagnetico}
Dal punto di vista sperimentale, sono evidenti in natura fenomeni di magnetizzazione spontanea in alcuni materiali di tranzione quali Fe, Co e Ni. I fenomeni di magnetizzazione persistono per temperature inferiori ad una certa temperatura critica (che pu\`o essere nell'ordine anche dei $10^3K$), oltre alla quale i materiali perdono totalmente le loro propriet\`a magnetiche. Allo stesso modo, la suscettivit\`a magnetica, mostra anche lei un comportamento divergente nei pressi della temperatura critica ($T_C$). Materiali che presentano questo tipo di comportamento si dice che presentano una \textit{transizione di fase del second'ordine}. L'evidenza di comportamenti magnetici da parte di alcuni materiali, in totale assenza di campo magnetico esterno, suggerisce che i dipoli magnetici atomici hanno un certo grado di allineamento ordinato in una certa collettivit\`a, meglio noto come \textit{ordinamento spontaneo}. Il Ferromagnetismo \`e il fenomeno  pi\`u diffuso di ordinamento magnetico, con rottura spontanea della simmetria spaziale del sistema in cui il parametro d'ordine \`e rappresentato dalla magnetizzazione. Lo stabilirsi di un ordine, suggerisce che ci sia un certo tipo di interazione cooperativa tra i vari spin. Dal punto di vista microscopico, questo tipo di iterazione \`e possibile considerarla ristretta solo agli spin \textit{primi vicini}, generalizzando la pi\`u restrittiva condizione di spin indipendenti.

La fisica moderna, circa i fenomeni di ordinamento magnetico, attribuisce un'effettiva interazione tra gli spin. Non si st\`a aggiungendo una "forza" alle quattro forze basi dell'universo. L'interazione tra spin e' di tipo magnetico, solo che la descrizione diventa pi\`u semplice se si introduce una forza di interazione tra gli spin. In questo modo si pu\`o passare anche a definire il concetto di \textit{onde di spin}. Quantizzando il campo di spin, \`e possibile introdurre una quasiparticella che rappresenta il mediatore dell'interazione di spin, chiamata \textit{magnone}. I magnoni sono bosoni e possono essere rilevati sperimentalmente tramite scattering anaelastico di neutroni (o di particelle comunque pesanti, prive di cariche, con un proprio dipolo magnetico permanente).
\subsection{Modello di Ordinamento Magnetico di Heisenberg}
Si consideri un sistema ferromagnetico, di ioni o atomi aventi un dipolo magnetico intrinseco di intensit\`a $\mu = g_L\mu_BJ$ diretto lungo il proprio asse $z$. Sotto l'assunzione che il sistema sia vicino al completo allineamento di tutti gli spin \`e semplice valutare la magnetizzazione che pu\`o essere nell'ordine di un minimo di $0.1T$ ad un massimo di $1T$, valore che \`e stato possibile anche verificare sperimentalmente. Con una suscettivit\`a paramagnetica nell'ordine di $\chi \sim 10^{-3}$, questo vuol dire che, per avere un allineamento spontaneo degli spin, a temperatura ambiente, \`e necessario che sia presente, all'interno del mezzo, un campo magnetico grandissimo nell'ordine dei $10^2T - 10^3T$. Questo ipotetico campo, venne chiamato \textit{campo molecolare} nelle prime teorie sul ferromagnetismo di inizio secolo '900. 
Usando gli strumenti della meccanica classica, non conoscendo altro viene spontaneo scrivere il potenziale di interazione tra gli spin come interazione tra due dipoli magnetici classici, entrambi di intensit\`a $\mu=g_L\mu_BJ$
\newl{U = \frac{1}{r^3}\left[\mu_1\cdot\mu_2-3\frac{(\mu_1\cdot r)(\mu_2\cdot r)}{r^2}\right].}
Assumendo la distanza atomica nell'ordine di $2\cdot10^{-8}cm$ risulta che $U\sim10^{-5} eV$ che \`e circa tre ordini di grandezza pi\`u piccolo. Questa energia, corrisponde alla temperatura di $1K$, quindi secondo la teoria classica, non dovrebbere assolutamente esistere fenomeni di magnetizzazione spontanea a temperatura ambiente, tutto dovrebbe essere "nascosto" dalle fluttuazioni date dall'agitazione termica. A questo punto \`e chiaro che si rende necessaria l'introduzione di un nuovo tipo di interazione di dipolo, interazione che non avr\`a nulla di classico ma sar\`a puramente quantistica. Le energie di interazione possono derivare solo da forze elettrostatiche o Coulombiane tra elettroni appartenenti alle shell pi\`u esterne degli atomi primi vicini. Dagli studi condotti sull'atomo di Elio (il caso pi\`u semplice in cui \`e possibile scontrarsi col problema di pi\`u elettroni) che conducono alla conclusione che la funzione d'onda multi-elettronica deve essere antisimmetrica, questo da un contributo energetico nell'ordine dell'$eV$, comunemente chiamata \textit{energia di scambio} la quale rompe la degenerazione tra gli stati aventi differenti autovalori dello spin totale $S = \sum S_i$. L'energia di scambio, a volte interpretata come evidenza dell'esistenza della forza di interazione di scambio, nasce dal contributo energetico della parte repulsiva del potenziale tra gli elettroni, e non ha nessun tipo di analogo classico. Non pu\`o essere interpretato come interazione delle nuvole elettroniche.
Prendiamo come esempio una funzione d'onda a pi\`u elettroni 
\newl{\psi^+_{A,B} = \left[\phi_A(r_1)\phi_B(r_2) \pm \phi_A(r_2)\phi_B(r_1)\right].}
L'energia di interazione tra i due elettroni \`e data da
\newl{E&=&\left\langle\psi^{\pm}\left|\frac{e^2}{|r_1-r_2|}\right|\psi^{\pm}\right\rangle = \int\,d^3r_1\,d^3r_2 |\phi_A(r_1)|^2|\phi_B(r_2)|^2 \frac{e^2}{|r_1-r_2|} \pm  \nonumber \\ 
       &\pm& \int\, d^3r_1\,d^3r_2 \phi_A(r_1)^*\phi_B(r_2)^* \frac{e^2}{|r_1-r_2|} \phi_A(r_1)\phi_B(r_2)  }
come si pu\`o notare il contributo di scambio, puramente di interpretazione quantistica, pu\`o essere sia positivo che negativo. La prima parte \`e data dall'interazione elettrostatica a cui si aggiunge la seconda parte detta appunto di \textit{scambio} che \`e differente per configurazioni diverse di spin (paralleli o antiparalleli). Nell'atomo di Elio ho i due stati di singoletto e tripetto a cui sono relativamente associate $E_S$ ed $E_T$ che rappresentano il contributo dell'energia di scambio dello stato di singoletto e di tripletto. Possiamo quindi definire una Hamiltoniana efficace introducendo una fittizza \textit{interazione tra spin} scalata da un certo parametro $J = E_S-E_T$, 
\newl{\op{H} _S = \frac{1}{4}(E_S+3E_T) - J\,\op{S} _1 \cdot \op{S} _2  }
Generalizzando l'Hamiltoniana a tutti i sistemi a due elettroni, \`e possibile osservare che l'originale interazione elettrostatica, insieme al principio di Pauli, creano una diretta interazione tra gli spin (non di tipo magnetico!) adatta ad assegnare differenti energie ai due tipi di configurazioni (spin paralleli e antiparalleli). Generalizzando il tutto al caso per sistemi ad $N$ elettroni il modello diventa lineare in quanto le interazioni, seppur Coulombiane, sono cmq da considerarsi a corto range (scermatura di Thomas-Fermi). Ridefinendo l'energia di punto zero a $(E_s+3E_T)$ possiamo scrivere l'\textbf{Hamiltoniana di Haisenberg} come
\newl{\op{H} _S = -\frac{1}{2}\sum_{<i,j>}J_{i,j}\op{S} _1 \cdot \op{S} _2,}
dove gli indici $i,j$ corrono solo sui primi vicini e l'$1/2$ davanti a tutto c'\`e solo per eliminare i doppi contributi. Questa Hamiltoniana \`e lo strumento base per alcune importanti teorie sull'ordinamento magnetico. Alcune cose di particolare interesse sono le seguenti:
\begin{itemize}
	\item Gli \textit{spins} di cui si parla dovrebbero appunto essere il momento angolare totale dei vari atomi o ioni che vengono considerati, ma in molti casi il fenomeno del \textit{quenching orbitale}, precedentemente discusso, fa s\`i che non ci sia una reale distinzione e in pratica \`e comodo parlare in termini di operatori di spin.
	\item Le coppie interagenti possono essere scelte in funzione alla struttura del cristallo. In alcuni casi pu\`o essere oppurtuno considerare anche interazioni a secondi vicini ecc ...
	\item La costante di accoppiamento viene solitamente determinata in modo sperimentale. Solitamente in media si ha che $J_{i,j} = J_{j,i}$ e $J_{i,i} = 0$. Con $J>0$ il ground state dell'Hamiltoniana rappresenta i fenomeni ferromagnetici, per $J<0$ rappresenta i fenomeni antiferromagnetici.
	\item $\op{H} _S$ ha completa simmetria rotazionale, quindi non c'\`e a priori una direzione preferenziale di allineamento per gli spin del sistema. Ci sono delle differenze rispetto al modello $XY$, in cui gli spin sono confinati in un piano. Il modello di Ising rappresenta un "giochino" molto divertente in cui applicare i metodi della meccanica statistica e ottenere risultati numerici sulle transizioni di fase.
	\item Usando la completa simmetria rotazionale e i metodi di calcolo delle medie della meccanica statistica \`e possibile verificare che i valori di aspettazione per la magnetizzazione sono esattamente zero. Ma questo non si osserva in natura, la spiegazione \`e contenuta nel fatto che questo sistema hamiltoniano \`e sottoposto ad una rottura spontanea della simmetria rotazionale, anche per campi magnetici esterni infinitesimali. Le medie statistiche non sono degli invarianti rotazionali, quindi una fluttuazione nel campo magnetico (anche piccolissima) pu\`o determinare una complessiva e spontanea magnetizzazione di tutto il sistema. 
	\item L'interazione magnetica, descritta dalla costante di accoppiamento $J_{i,j}$, deriva dal \textit{diretto scambio} ...[non si capisce bene]
\end{itemize}
\subsection{Teoria di Campo Medio del Ferromagnetismo}
Come \`e stato discusso precedentemente la materia condensata, in alcune situazioni presenta delle macroscopiche caratteristiche magnetiche che sono riconducibili a fenomeni di ordinamento spontaneo dei dipoli magnetici (o \textit{spin} ) all'interno dei reticoli dove sono confinati. Questo ordinamento dei dipoli magnetici, potrebbe essere sostenuto da un ipotetico campo magnetico molto intenso che abbiamo chiamato \textit{campo molecolare} $H_m$. Questo ipotetico campo molecare dovrebbe esiste indipendentemente dalla presenza o meno di un campo magnetico esterno $H_0$. Assumento il caso pi\`u generale in assoluto, in caso ci sia pure un campo magnetico esterno, possiamo dire che su ogni particella agisce un campo totale dato dalla somma dei due contributi $H_T = H_0 + H_m$. La fondamentale ipotesi, fatta da Weiss nel 1901, verte sul fatto che esiste una proporzionali\`a diretta dal campo molecolare e la magnetizzazione presentata dal materiale
\newl{ {\vet H} _m = w {\vet M}, }
dove $w$ \`e la costante di Weiss (adimensionale nelle unit\`a CGS) e contiene un po' tutta la fisica del problema. La stima di $w$ \`e di circa $w\sim10^3, 10^4$, con questa (e anche altre assunzioni) Weiss svilupp\`o la \textbf{teoria di campo medio del ferromagnetismo}, che aveva una discreta attendibilit\`a ma partiva da una spiegazione fisica dell'origine completamente sbagliata. Ora si conoscono in modo pi\`u preciso le origini del ferromagnetismo, comportamento dovuto all'energia di scambio, spiegato nel modello di Heisenberg, dunque riformuleremo una teoria di campo medio proprio partendo da questo aspetto. In presenz di un campo magnetico esterno, l'Hamiltoniana di Heisenberg \`e completata dall'inserimento delle relative energie di interazione tra gli spine possiamo riscriverla come
\newl{\op{H} _S = -\frac{1}{2} \sum_{<i,j>}J_{i,j} \op{S} _i \cdot \op{S} _j  -g \mu_B {\vet H} _0 \cdot \sum_j \op{S} _j ,}
dove g rappresenta il fattore giromagnetico. Una soluzione esatta dell'Hamiltoniana di Hesienberg, non \`e ancora conosciuta, a causa delle difficolt\`a di districare il complesso accoppiamento tra gli spin. La tessa difficilt\`a si trova nel calcolare la funzione di partizione del sistema, che richiede una somma su tutte le possibili configurazini. A qeusto punto, possiamo considerare un'approssimazione di campo medio del sistema, cercando di derivare una Hamiltoniana rappresentativa delle sole interazioni \textit{locali} tra gli spin e un campo magnetico "medio". 

Si fissi l'attenzione sul sito reticolare $i$-esimo \footnote{Dinamica a singolo spin flip.} e assumiamo per semplicit\`a che $J_{i,j} = J$, l'Hamiltoniana che descrive la dinamica dello spin $S_i$ sar\`a
\newl{\op{H} _i = -S_i \cdot \left(J\sum_{j} S_j + g\mu_BH_0\right).}
Sostituendo agli spin il loro valore di aspettazione $S_j = \left\langle S_j \right\rangle = \left\langle S \right\rangle$ e dimenticando le correlazioni, quindi $\left\langle S_iS_j \right\rangle = \left\langle S_i \right\rangle \left\langle S_j \right\rangle$, \`e vero che si viene a perdere un po' di fisica in quanto ora siamo ritornati a considerare elettroni totalmente indipendenti, ma come si potr\`a vedere pi\`u avanti, le cose principali rimarranno. A questo punto \`e possibile scrivere l'Hamiltoniana di campo medio come
\newl{\op{H} _i^{MF} = -g\mu_B S_i \cdot H_{eff} }
dove l'Hamiltoniana efficace \`e rappresentata da
\newl{H_{eff} = H_0 + \frac{J}{\mu_B}Z\left\langle S \right\rangle }
dove $Z$ \`e il numero di primi vicini. Questa approssimazione \`e equivalente, come detto precedentemente, a non considerare pi\`u le correlazioni tra gli spin. In questo modo, eliminando i temini che consideravano le fluttuazioni degli spin che erano fondamentali per le transizioni di fase, questo modello perde di validit\`a vicino alla temperatura critica, quindi abbiamo reintrodotto uno schema di particelle indipendenti nella nostra Hamiltoniana a pi\`u particelle. La magnetizzazione di questo sistema di spin \`e dato dal valore di aspettazione dei momenti di dipolo nell'unit\`a di volume
\newl{M = \frac{N}{V} g\mu_B \left\langle S \right\rangle.}
Riprendendo l'ipotesi di Weiss dell'espressione del campo molecolare in funzione alla magnetizzazione \'e possibile scrivere
\newl{H_m = \frac{J}{g\mu_B} Z \left\langle S \right\rangle = wM}
da cui \`e possibile ricavare la costante di Weiss
\newl{w = \frac{V}{N} \frac{JZ}{g^2\mu_B^2} }
con l'Hamiltoniana che risulta $H_{eff} = H_0 + wM$. Queste relazioni mostrano che il campo agente sugli spin \`e autoconsistente (creato dagli stessi spin).
Ma ancora non \`e sufficiente per spiegare l'esistenza della magnetizzazione spontanea. A questo scopo ci si servir\`a dei metodi della meccanica statistica. Il sistema che si sta studiando \`e un insieme di dipoli indipendenti che interagiscono tutti con il campo $H_{eff}$, come descritto dall'Hamiltoniana paramagnetica (quella di campo medio). All'equilibrio termico la magnetizzazione \`e diretta lungo la stessa direzione del campo. Assumento $z$ la direzione del campo, si ottiene che 
\newl{M= M_0 \mathcal{B} _S \left(\frac{g\mu_B S H_{eff}}{k_BT}\right) =  M_0 \mathcal{B} _S \left(\frac{g\mu_B S }{k_BT} (H_0 + wM)\right)}
dove il numero quantico di spin $S$ rimpiazza $J$, $\mathcal{B}_s(x) $ rappresenta la funzione di Brillouin e $M_0$ e la magnetizzazione di saturazione definita come 
\newl{M_0 = \frac{N}{V} g \mu_B S. }
L'equazione per ma magnetizzazione scritta sopra, \`e una equazione implicita. Pu\`o essere comodo risolverla in modo grafico per avere idea del senso fisico di cui \`e pregnante. (Vedere appunti, in cui \`e disegnata bene). Per campo magnetico esterno posso vedere che l'intersezione, oltre ad essere quella con $M=0$ (sempre presente) c'\`e anche quella a $M\neq 0$, quindi questo dimostra l'esistenza della magnetizzazione spontanea. Pi\`u la temperatura si alza, maggiore \`e il coefficiente angolare. Nel momento in cui le due curve sono tangenti, quello \`e il punto in cui la temperatura ha raggiunto in punto critico. Quindi \`e stato possibile dimostrare che anche in approssimazione di campo medio, \`e ponderata la mangetizzazione spontanea ed \`e presente una transizione di fase. [Capire quando la teoria di campo medio smette di essere attendibile.]
\subsection{Transizioni di Fase - Ferromagnetismo}
Le transizioni da una fase ferromagnetica ad una fase normale, quando $T$ inizia a diventare molto pi\`u grande della $T_C$ si definiscono transizioni di fase del secondo ordine, in cui \`e presente la continuit\`a della quantiti\`a estensiva (come nel nostro caso la magnetizzazione), ma \`e presente una divergenza nella funzione di risposta di questa grandezza, che nel caso in studio \`e rappresentata dalla suscettivit\`a magnetica. Per temperature $T > T_C$ si ha che, in assenza di campo magnetico esterno, $M = 0$, mentre la suscettivit\`a magnetica obbedisce alla legge di Curie $\chi\propto 1/(T-T_C)$.

Quando $T < T_C$ magnetizzazione e suscettivit\`a possono essere ricavate dai risultati della sezione precedente. Di fondamentale importanza per\`o \`e sopratutto lo studio del sistema a temperatura prossima a quella critica. Si nota che intorno al punto critico, le quantit\`a come Magnetizzazione e Suscettivit\`a magnetica obbediscono ad una legge di potenza che per la magnetizzazione \`e rappresentata da
\newl{ M \propto (T_C -T)^{\beta},}
dove $\beta \sim 0.3$. La suscettivit\`a, avvicinandosi a $T_C$, diverge con la legge di potenza
\newl{\chi \propto  |T-T_C|^{\gamma}}
con $\gamma \sim  -1.3$. Una teoria microscopica della transizione da uno stato magnetizzato a uno non magnetizzato deve essere un grado di giustificare questi comportamenti divergenti. Nel caso della teoria in approssimazione di campo medio, sviluppata nella precedente sezione, \`e possibile calcolare questi \textit{esponenti critici} in modo semplice. Si consideri lo stato ordinato in cui $T < T_C$. La magnetizzazione \`e data da 
\newl{M = M_0 \mathcal{B} _S \left(\frac{g\mu_B S}{k_BT} (H_0+wM)\right)}
con $H_0=0$. \`E stato trovato che nel limite in cui $T\to T_C^-$ la magnetizzazione sparisce, in accordo con la funzione di Brillouin. Quindi possiamo espandere la funzione di Brillouin in un intorno di $T_C$ arrivando a scrivere che 
\newl{M \propto (T_C - T)^{1/2}.}
Quindi, in approssimazione di campo medio, la magnetizzazione, decresce avvicinandosi a $T_C$ con una legge di potenza, con esponente critico $1/2$.

In generale, anche in presenza di un campo magnetico esterno, il materiale continua a presentare un comportamento ferromagnetico per  $T<T_C$ e completamente paramegnetico per $T>T_C$.
Vicino alla temperatura critica, la magnetizzazione in questo caso \`e pi\`u piccola di $M_0$ perch\`e $\mu_BH_0 \ll k_BT$ come discusso precedentemente, quindi abbiamo nuovamente la possibilit\`a di espandere la formula della magnetizzazione, prendendo solo il primo termine dello sviluppo si arriva a scrivere
\newl{M = \frac{C}{T-T_C} H_0 ,}
dove $C$ \`e una costante. Ricavando l'equazione per la suscettivit\`a magnetica otteniamo la \textbf{Legge di Curie-Weiss}
\newl{\chi = C \cdot (T-T_C)^{-1}}
chiamata cos\`i a causa del fatto che viene appunto derivata dall'estensione della legge di Curie per i materiali ferromagnetici. Vicino a $T_C$ la suscettivit\`a diverge con una legge di potenza, con esponente critico pari a $-1$. 

In conclusione la teoria di campo medio per i fenomeni magnetici produce due esponenti critici pari a $\beta = 0.5$ e $\gamma=-1$, per un comportamento vicino alla temperatura critica. Questi valori sono apprezzabilmente differenti da quelli misurati in modo sperimentale. I valori misurati mostrano che la magnetizzazione descresce pi\`u rapidamente e la suscettivit\`a magnetica diverge pi\`u rapidamente, man mano che ci si approccia alla temperatura critica. Le discrepanze della trattazione a campo medio con i dati sperimentali, nell'intorno della temperatura critica, vanno cercate nel completo annullamento delle correlazioni tra spin nella teoria di campo medio. Queste correlazioni diventano importanti quando la dinamica di un sistema \`e determinata dalla piccole fluttuazioni che ha, in prossimit\`a della temperatura critica.

La teoria di campo medio fornisce comunque dei risultati utili, soprattutto per piccoli campi magnetici. In questi casi, \`e sempre possibile usare l'espansione di prima, ma troncata al terzo ordine. In questo modo ho che la magnetizzazione segue una legge di potenza
\newl{M \propto H_0^\delta}
con $\delta=1/3$. [Vedere il grafico a pag Mag-34.]
\subsection{Onde di Spin}
Come mostrato nella sezione precedente, la trattazione del ferromagnetismo in approssimazione di campo medio, per $T\to T_C$ perde di validit\`a. Un approccio pi\`u raffinato che permette di ottenere un modello pi\`u generale, \`e quello in cui si tiene conto del contributo energetico di ogni spin e delle varie correlazioni. In altre parole, per una corretta valutazione statistica della funzione di partizione, si rende necessaria la conoscenza dello spettro delle energia del sistema nel suo ground state. \`E necessario introdurre una teoria quantistica elementare dell'eccitazione per sistemi di spin. Il punto di partenza \`e l'Hamiltoniana di Heisenberg
\newl{\op{H} _S = \frac{1}{2}\sum_{<i,j>}J_{i,j}\op{S} _i \cdot \op{S} _j -g\mu_B{\vet H_0} \cdot \sum_j\op{S} _j .}
Assumiamo che \textit{gli effetti dell'interazione di scambio}, siano sensibili solo tra spin primi vicini, con una costante di accoppiamento uguale e positiva per gli spin di tutto il reticolo $J_{i,j} = J > 0$ per tutti gli $i,j$. In un cristallo di $N$ spin \textit{uguali}, con numero quantico di spin $S$, in cui ogni spin abbia un numero di primi vicini pari a $z$ quindi il numero di termini della somma \`e $Nz$. Il \textit{ground state} del sistema si realizza quando tutti gli spin sono allineati lungo l'asse $z$, quindi il singolo autovalore sar\`a $\left\langle S_{j,z} \right\rangle  = m_j = S$ e la corrispondende funzione d'onda sar\`a  $\ket{NS} = \prod_j \ket{S,m_j=S}  $ . Questa funzione d'onda, proprio per come \`e stata costruita, ha un autostato di energia
\newl{E_0 = -g\mu_B H_0 NS - \frac{Nz}{2}JS^2. }
Definisco gli operatori di innalzamento e abbassamento del mio sistema di spin come
\newl{\op{S} _j ^\pm = S_{jx}\pm i S_{jy}}
cone le loro regole di commutazione $[S_i^+,S_i^-] = 2S_{iz}\delta_{i,j}$ e $[S_{iz},S_j^\pm] = \pm S_j^\pm\delta_{i,j}$.
Sostituisco nell'Hamiltoniana di Heisenberg gli operatori appena definiti
\newl{\op{H} _S = \frac{J}{2} \sum_{<i,j>} \left[S_{i,z} S_{j,z} + S_i^-S_j^+\right] - g\mu_BH_0\sum_j S_{j,z}  \label{HAM:HEI:OP}}
Una \textbf{teoria elementare dell'eccitazione} consiste nello studiare la transizione da uno stato completamente allineato, in cui solo uno spin compie un completo rovesciamento, facendo passare $m_j=S$ a $m_j = S-1$. La variazione di energia causata da un singolo spin flip \`e di
\newl{\Delta E = E_1 - E_0 = g\mu_BH_0 + zJS}
\`E semplice verificare che l'esatta variazione di energia arriva considerando solo i termini sulla diagonale nell'Hamiltoniana (\ref{HAM:HEI:OP}). 
Per trovare la funzione d'onda di tutti gli stati eccitati potrei usare l'operatore di abbassamento in questo modo
\newl{\ket{\psi_j} = S_j^- \ket{NS} = \sqrt{2S} \ket{m_j = S-1} \cdot \prod_{i\neq j} \ket{m_i = S} ,  }
ma questa funzione d'onda non \`e autostato dell'Hamiltoniana totale. Ho dei termini fuori diagonale che  danno un contributo
\newl{S_i^- S_j^+ \ket{\psi_k} = 2S\delta _{j,k} \psi_i ,}
dove l'eccitazione \`e trasferita sui primi vicini al sito $i-esimo$ e non abbiamo una equazione agli autostati da risolvere. L'operazione, ripetuta per l'Hamiltoniana di Heisenberg, corrisponde a scambiare le interazioni tra gli spin. Questo suggerisce un comportamento peculiare, grazie alla propriet\`a di \textit{spin indistinguibili}, una singola eccitazione \`e totalmente delocalizzata e non possiamo restringerla ad uno spin ben localizzato, si ha quindi la probabilit\`a di avere eccitazione in ogni posizione del reticolo, non solo in una posizione ben localizzata.
Per trovare i corretti autostati dell'Hamiltoniana, come appena visto, \`e necessario costruire una funzione d'onda, come combinazione lineare di stati eccitati. Con $R_n$ si identifica la posizione dello spin nel reticolo, mentre $R_n+\delta$ rappresenta tutti i primi vicini, con $\delta$ che pu\`o variare di volta in volta in funzione alla geometria del cristallo.
Verifico che \`e un effettivo autostato inserendolo nell'Hamiltoniana di partenza. In questo modo, ottengo una serie di autostati di energia
\newl{E_{k} = E_0 + zJS + g\mu_BH_0 - JS\sum_{\delta} e^{i k\cdot \delta} = \hbar \omega_{k_0} \label{DISP:EN}}
questa relazione di dispersione rappresenta lo spettro delle \textit{onde di spin}. \`E possibile riscrivere la relazione di dispersione in modo pi\`u utile per i successivi conti definendo il coefficiente $\gamma_k = 1/z \sum_\delta e^{ik\cdot\delta}$
\newl{\Delta E_k = g\mu_BH_0 + zJS(1-\gamma_k) = \hbar\omega_k \label{DISP:GAMMA}}

L'eccitazione non \`e localizzata, infatti calcolando la correlazione tra spin si ottiene
\newl{\bra{\psi_k} S_{R_nx} S_{R_mx} + S_{R_ny} S_{R_my} \ket{\psi_k} = \frac{2S}{N} \cos [k\cdot (R_n-R_m)] }
una correlazione sinusoidale. Se calcolo  $<S_{iz}> = S - 1/N$, questo risultato conferma nuovamente che il flip anche solo di uno spin si distribuisce su tutto il reticolo, mentre $<S_{ix}> =0$.

Ricordando quanto detto inizialmente, l'eccitazione non pu\`o essere locale, ma si propaga su tutto il reticolo e lo fa tramite la legge di dispersione (\ref{DISP:EN}) che rappresenta l'elementare eccitazione di un sistema di spin. L'idea molto potente \`e quella di applicare lo stesso formalismo usato per trattare le vibrazioni di reticolo, anche alle onde di spin. Nella teoria delle vibrazioni reticolari, i punti reticolari condividono le oscillazioni termiche organizzandosi in un moto armonico comune con una sua ben nota relazione di dispersione. Nel presente cristallo di spin l'eccitazione magnetica, dovuta al singolo disallineamento di uno spin, viene condivisa da tutti gli spin del reticolo. \`E quindi possibile mostrare che la componenete trasversa dei vettori di spin $S_{j,x}$ e $S_{j,y}$ della funzione d'onda $\ket{\psi_k} $, tra i vari spin primi vicini, differisce solo per un fattore di fase costante, esattamente come succede per le oscillazioni di reticolo. La dinamica delle eccitazioni elementari deve essere trattata con il formalismo della seconda quantizzazione tramite operatori di creazione e distruzione che aggiungono o tolgono la relativa quasi-particella che provoca l'eccitazione. In questo modo \`e possibile parlare di \textbf{fotoni} che rappresentano la quantizzazione del campo elettromagnetico, di \textbf{fononi} per la quantizzazione del campo di oscillazioni di reticolo e infine di \textbf{magnoni} come quantizzazione delle eccitazioni magnetiche di un sistema di spin. La rappresentazione statistica delle onde di eccitazione magnetica in un cristallo di spin lo si esegue pensando alla statistica di un \textbf{gas di magnoni}, mentre per ora ci si \`e limitati a trattare eccitazioni di singolo spin. Questi sono stati eccitati di energia $\hbar\omega_{k_0}$. I livelli energetici per stati con $2,3,4,...$ eccitazioni di single spin possono essere calcolati solo in modo approssimato a cusa della non linearit\`a dell'Hamiltoniana. Per esempio una doppia eccitazione pu\`o essere costruita come
\newl{\ket{\psi_{k,h} } = \sum_{R_m,R_n}C_{R_m,R_n}(k,h) S_{R_m}^- S_{R_n}^- \ket{NS}  ,}
e non \`e per nulla semplice rappresentarlo come somme di onde di spin. Questo fatto pu\`o essere interpretato come la prova dell'esistenza di due tipi di interazione tra le onde di spin.
Una prima interazione dinamica in cui \`e pi\`u probabile che la seconda eccitazione si origini vicino al sito gi\`a eccitato, questo perch\`e farebbe diminuire l'energia del sistema. La coppia di eccitazioni si propaga nel cristallo tramite due onde di spin di energia inferiore a $\hbar(\omega_k+\omega_k)$. In questo modo si crea uno stato legato a causa dell'interazione attrattiva che si viene a formare. Una interazione cinematica \`e meno probabile. Da vita ad una interazione repulsiva. Queste interazioni sono il risultato della originale definizione di onde di spin, come interazioni di eccitazione nel ground state di un sistema a comportamento ferromagnetico [ma non ha senso!!!!!]. Una volta che si ha un magnone, un secondo magnone non pu\`o essere creato nello stesso identico background ferromagnetico. La stima corretta del contributo energetico delle \textit{eccitazioni multiple} per poter determinare le propriet\`a statistiche del sistema \`e rilevante solo per alte temperature, quando la densit\`a del numero di eccitazioni \`e relativamente alto.

Lo spettro di eccitazione di un sistema magnetico ordinato, pu\`o essere indagato sperimentalmente usando il metodo dello \textit{scattering anaelastico di neutroni}. Neutroni termici vengono diffratti dall'interazione sia coi fononi che con i magnoni, rendendo impossibile separare i due contributi, magnonici e fononici. Se i neutroni sono spin-polarizzati, \`e possibile distinguere il contributo dei \textit{magnoni} perch\`e solo i neutroni, diffratti con spin-flippato, avranno creato o assorbito un magnone. Dalla posizione e dalla larghezza dei picchi nella sezione d'urto differenziale dei neutroni con spin-flippato \`e possibile determinare lo spettro dei magnonie il loro rispettivo tempo di vita. Quest'ultima grandezza \`e direttamente connessa all'interazione tra onde di spin.

Per concentrarsi sullo studio delle onde di spin si consideri un gas di magnoni a bassa temperatura. In questo caso solo le onde di spin con energia pi\`u bassa (corrispondenti a vettori d'onda pi\`u piccoli) possono essere eccitate. In questo modo \`e possibile espandere il coefficienti $\gamma_k$ dell'eqauazione (\ref{DISP:GAMMA}) e usando la simmetria traslazionale di un reticolo la somma lineare da zero e si ha che
\newl{\gamma_k = 1 - \frac{1}{2z}\sum_{\delta}\left(k\cdot\delta\right)^2 + \cdots   }
da cui si ottiene una relazione di dispersione quadratica
\newl{\hbar\omega_k = g\mu_BH_0 + \mathcal{D}\cdot k^2}
dove, per esempio, la costante $\mathcal{D}$ per un semplice cristallo cubico \`e uguale alla costante $a$ del reticolo. La forma quadratica \`e quella tipica della particella libera. In assenza di campo magnetico esterno si ha dunque un \textit{gas di magnoni liberi non interagenti} in cui sono molto popolati gli stati a lunghezza d'onda lunga, con energia vicina a quella di ground state. L'esistenza di questo tipo di interazione aumenta la decrescita della magnetizzazione (all'aumentare di $T$) rispetto ai risultati trovati con la teoria di campo medio (cio\`e si ha una decrescita maggiore rispetto alla funzione di Brillouin), perch\`e l'attivazione termica delle onde di spin abbassa complessivamente la magnetizzazione. 
In questo contesto, la statistica del gas di magnoni liberi \`e quella di un \textit{gas di bosoni} con potenziale chimico nullo perch\`e il numero di magnoni non \`e fissato, come succede per i fotoni e i fononi. La densit\`a degli stati \`e quella di un sistema $3D$ di particelle liberi non interagenti con energia $E\propto k^2$ e $g_{3D} \propto \sqrt{E}$. Ora \`e possibile passare a calcolare la variazione di magnetizzazione spontanea come funzione della temperatura. Per ogni eccitazione $(S\to S-1)$ la magnetizzazione descresce di una quantit\`a pari a $g\mu_B/V$, rispetto alla magnetizzazione di saturazione $M_0= (N/V)g\mu_B S$ la variazione \`e 
\newl{\frac{\Delta M}{M_0} = \frac{M_0-M}{M_0} = \frac{1}{NS} \sum_kn_k}
dove $n_k$ \`e il numero di occupazione media dello stato con $k$. Sommando su tutti i possibili stati, la somma diventa un integrale e usando la distribuzione di Bose-Einstein come funzione di densit\`a di occupazioni degli stati (i magnoni, come detto precedentemente sono dei "quasi-bosoni") \`e possibile rivalutare la variazione di magnetizzazione rispetto alla magnetizzazione di saturazione
\newl{\frac{\Delta M}{M_0} \propto \int_0^{+\infty}\, dE\,g_{3D}(E) \frac{1}{e^{e/K_BT} - 1 } = \int_0^{+\infty}\, d\omega \, \frac{\sqrt{\hbar\omega}}{e^{\hbar\omega/k_BT} -1} \sim (k_BT)^{3/2}.   } 
Come si pu\`o notare, quindi, a basse temperature, la magnetizzazione decresce con una legge di potenza di $T^{3/2}$ in particolare
\newl{\boxed{M(T) = M_0 (1-cT^{3/2}), }}
dove $c$ \`e una costante. Questo comportamento \`e confermato sperimentalmente. A questo punto l'energia totale del gas di magnoni liberi non interagenti \`e facile calcolarla
\newl{E \propto \sum_k \Delta E_k n_k \propto \int_0^{+\infty}\, d\omega \, \frac{\hbar \omega (\hbar \omega)^{1/2}}{e^{\hbar\omega/K_BT} - 1} \propto (k_BT)^{5/2}}
da cui segue che il calore specifico \`e proporzionale a $T^{3/2}$ risultato noto come \textit{Legge di Bloch}. In uno stato ferromangetico a bassa temperatura questo calore specificoda un contributo rilevante al calore specifico del reticolo totale (da cui segue la teoria di Debay da cui si ricava una dipendenza del calore specifico ti tipo $T^3$). 

Il calolo della magnetizzazione spontanea, in funzione alla temperatura, nel modello di Heisenberg, \`e di particolare interesse, in quanto permette di associare il fenomeno delle transizioni di fase alla dimensionalit\`a del problema in studio. Per un sistema $1D$ la funzione di degenerazioen degli stati \`e proporzionale a $g_{1D} = 1/\sqrt{E}$, mentre per un sistema $2D$ la $g_{2D}$ \`e una costante, questo rende chiaro il fatto che tutti gli integrali scritti in precedenza, per sistemi $1D$ o $2D$ per $\omega\to0$, divergono. Dato che fisicamente non \`e sensato un risultato in cui si ha magnetizzazione infinita, questo fatto viene interpretato come \textit{assenza di magnetizzazione spontanea per sistemi "basso" dimensionali}. Le fluttuazioni termiche a bassa dimensionalit\`a, distruggono le correlazioni tra i vari spin impedendo che ci sia un effetto a lungo range come succede invece nei sistemi ad alta dimensionalit\`a. Si noti che questo NON \`e il caso del modello di Ising, in cui gli spin sono considerati con la forzatura di essere allineati lungo l'asse $z$, e rende quindi l'Hamiltoniana di Ising simmetrica. Questa propriet\`a  \`e comune in un condensato di Bose-Einstein di particelle bosoniche, ed \`e diretta conseguenza del teorema di Mermin-Wagner, legato direttamente alla rottura di simmetria della dimensionalit\`a del sistema.

Si vuole concludere questo capitolo molto vasto, con  un paio di commenti circa il comportamento bosonico del gas di magnoni liberi non interagenti:
\begin{itemize}
	\item Nella fisica delle particelle e della materia condensata il teorema di Goldstone afferma che ad ogni rottura spontanea di simmetria , una nuova particella scalare priva di massa appare nello spettro delle possibili eccitazioni. Quindi esiste sempre una particella scalare (\textit{Bosone di Goldstone})  per ogni rottura di simmetria. Nei fenomeni di ordinamento magnetico l'originale simmetria rotazionale dell'Hamiltoniana di Heisenberg, viene spontaneamente rotta localmente da uno spin che eccita il sistema. In questo caso, i bosoni di Goldstone che si vengono a generare dalla rottura della simmetria rotazionale sono appunto i \textit{magnoni}. Le onde di spin, possono essere viste come un caso particolare di un pi\`u generale contesto fisico.

	\item L'algebra $SU(2)$ degli operatori di spin, con numero quantico $S$, pu\`o essere mappata in una singola oscillazione bosonica, come dimostrato dalle trasformazioni di Holstein-Primakoff. In modo particolare, gli operatori di innalzamento e abbassamento degli stati, risultano proporzionali a quelli di creazione e distruzione. Questi operatori hanno regole di commutazione bosonica\footnote{Nel senso che le eccitazioni che sono state trattate in questa sezione, sono caratterizzate da quasi-particelle dal comportamento bosonico. Il magnone \`e un bosone quindi passando al formalismo della seconda quantizzazione, gli operatori di creazione e distruzione di un magnone, dovranno avere le propriet\`a di creare e distruggere bosoni nello spazio di Fock. Gli stati iniziali di Slayter devono essere tutti simmetrizzati quindi le regole che legano gli operatori di creazione e distruzione, in questo ambito, sono regole di commutazione. Fossero stati fermioni, si avrebbe avuto una regola di anticommutazione con il costrain che $C_{\pm\ket{u} }^2  = 0$.} 
con il constrain che gli stati fisici accessibili sono caratterizzati da un numero quantico di occupazione che varia da $0$ a $n = 2S$. Le trasformazioni di H-P sono usate per convertire, con alcune approssimazioni, l'Hamiltoniana di Heisenberg nell'Hamiltoniana di onde di spin in seconda quantizzazione.
\end{itemize}



